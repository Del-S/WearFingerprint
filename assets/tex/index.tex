%% LyX 1.5.5 created this file.  For more info, see http://www.lyx.org/.
%% Do not edit unless you really know what you are doing.
\documentclass[a4paper,english,english,openright,cleardoubleempty,BCOR10mm,DIV11,12pt]{scrreprt}
\usepackage[T1]{fontenc}
\usepackage[utf8]{inputenc}
\usepackage[english]{babel}
\usepackage{array}
\usepackage{float}
\usepackage{longtable}
\usepackage{varioref}
\usepackage{wrapfig}
\usepackage{fancybox}
\usepackage{calc}
\usepackage{framed}
\usepackage{url}
\def\UrlBreaks{\do\/\do-}
\usepackage{graphicx}
\usepackage{placeins} %floatbarrier \FloatBarrier
%\usepackage{listing}
\usepackage{pdfpages}
\usepackage{epstopdf}
\usepackage[left=3cm,top=2.5cm,right=2.5cm,bottom=2.5cm]{geometry}
\usepackage{breakurl}
\usepackage{indentfirst}

% Cut words in half
%\hyphenation{Word-cut}

\makeatletter

\usepackage[font=small,labelfont=bf]{caption} %captiony

%%%%%%%%%%%%%%%%%%%%%%%%%%%%%% LyX specific LaTeX commands.
\providecommand{\LyX}{L\kern-.1667em\lower.25em\hbox{Y}\kern-.125emX\@}
\newcommand{\lyxline}[1][1pt]{%
  \par\noindent%
  \rule[.5ex]{\linewidth}{#1}\par}
\newcommand{\noun}[1]{\textsc{#1}}
%% Special footnote code from the package 'stblftnt.sty'
%% Author: Robin Fairbairns -- Last revised Dec 13 1996
\let\SF@@footnote\footnote
\def\footnote{\ifx\protect\@typeset@protect
    \expandafter\SF@@footnote
  \else
    \expandafter\SF@gobble@opt
  \fi
}

\renewcommand{\baselinestretch}{1.5} %line height

\expandafter\def\csname SF@gobble@opt \endcsname{\@ifnextchar[%]
  \SF@gobble@twobracket
  \@gobble
}
\edef\SF@gobble@opt{\noexpand\protect
  \expandafter\noexpand\csname SF@gobble@opt \endcsname}
\def\SF@gobble@twobracket[#1]#2{}
%% Because html converters don't know tabularnewline
\providecommand{\tabularnewline}{\\}

%%%%%%%%%%%%%%%%%%%%%%%%%%%%%% Textclass specific LaTeX commands.
\newenvironment{lyxcode}
{\begin{list}{}{
\setlength{\rightmargin}{\leftmargin}
\setlength{\listparindent}{0pt}% needed for AMS classes
\raggedright
\setlength{\itemsep}{0pt}
\setlength{\parsep}{0pt}
\normalfont\ttfamily}%
 \item[]}
{\end{list}}

%%%%%%%%%%%%%%%%%%%%%%%%%%%%%% User specified LaTeX commands.
%<-------------------------------společná nastavení------------------------------>
%s\usepackage[]{babel}%počeštění názvů (Obsah, Kapitola, Literatura atp.)
\usepackage[]{hyperref} %odkazy v  pdf jsou klikací s barevnými rámečky
\usepackage[numbers,sort&compress]{natbib} %balíček pro citace literatury
%\usepackage{hypernat}%interakce mezi hyperref a natbib
%\newcommand{\BibTeX}{{\sc Bib}\TeX}%BibTeX logo
\hypersetup{   % Nastavení polí PDF dokumentu
pdftitle={Radio Fingerprint Acquisition Using a SmartWatch},%
pdfauthor={Bc. David Sucharda},%
pdfsubject={},%
pdfkeywords={Fingerprint, Android, Wear}%
}
\usepackage{multicol}




%<-----------------------------volání stylů----------------------------------------->
% (znak % je označení komentáře: co je za ním, není aktivní)
%<------------------------------------písmo----------------------------------------->
%\usepackage{pkg/bc-latinmodern}
%\usepackage{pkg/bc-times}
\usepackage{pkg/bc-palatino}
%\usepackage{pkg/bc-iwona}
%\usepackage{pkg/bc-helvetika}


%<------------------------------záhlaví stránek------------------------------------>
%\usepackage{pkg/bc-headings}
\usepackage{pkg/bc-fancyhdr}

%<------------------------------hlavičky kapitol------------------------------------>
%\usepackage{pkg/bc-neueskapitel}
%\usepackage{pkg/bc-fancychap}

\makeatother

\usepackage{babel}

%java code block%

\usepackage{listing}
\usepackage{listings}
\usepackage{color}

\definecolor{dkgreen}{rgb}{0,0.6,0}
\definecolor{gray}{rgb}{0.5,0.5,0.5}
\definecolor{mauve}{rgb}{0.58,0,0.82}

\renewcommand*{\lstlistingname}{Code example} %prejmenovani
\renewcommand*{\lstlistlistingname}{List of code examples}

% syntax highlight pro jazyk Java %
\lstset{
  %frame=r,
  captionpos=b,
  language=Java,
  aboveskip=3mm,
  belowskip=3mm,
  xleftmargin=0.2mm,
  showstringspaces=false,
  columns=flexible,
  basicstyle={\small\ttfamily},
  numbers=none,
  numberstyle=\tiny\color{gray},
  keywordstyle=\color{blue},
  commentstyle=\color{dkgreen},
  stringstyle=\color{mauve},
  breaklines=true,
  breakatwhitespace=true,
  tabsize=3,
    inputencoding=utf8,
    extendedchars=true,
    literate=%
    {á}{{\'a}}1
    {č}{{\v{c}}}1
    {ď}{{\v{d}}}1
    {é}{{\'e}}1
    {ě}{{\v{e}}}1
    {í}{{\'i}}1
    {ň}{{\v{n}}}1
    {ó}{{\'o}}1
    {ř}{{\v{r}}}1
    {š}{{\v{s}}}1
    {ť}{{\v{t}}}1
    {ú}{{\'u}}1
    {ů}{{\r{u}}}1
    {ý}{{\'y}}1
    {ž}{{\v{z}}}1
    {Á}{{\'A}}1
    {Č}{{\v{C}}}1
    {Ď}{{\v{D}}}1
    {É}{{\'E}}1
    {Ě}{{\v{E}}}1
    {Í}{{\'I}}1
    {Ň}{{\v{N}}}1
    {Ó}{{\'O}}1
    {Ř}{{\v{R}}}1
    {Š}{{\v{S}}}1
    {Ť}{{\v{T}}}1
    {Ú}{{\'U}}1
    {Ů}{{\r{U}}}1
    {Ý}{{\'Y}}1
    {Ž}{{\v{Z}}}1
}

\begin{document}
\renewcommand{\figurename}{Picture}
\renewcommand{\tablename}{Table}
\renewcommand{\contentsname}{Content}
\renewcommand{\bibname}{Literature}
\renewcommand{\listfigurename}{List of pictures}
\renewcommand{\listtablename}{List of tables}
%~\thispagestyle{empty}{\small ~\vfill{}
%}{\small \par}

%~\thispagestyle{empty}\vfill{}
%Tato stránka je tzv. protititul a je graficky součástí titulní stránky.
%Nechte ji prázdnou, nebo na ni umístěte vhodnou fotografii či ilustraci.

\cleardoublepage{}~\thispagestyle{empty}\begin{center}\pagenumbering{roman}\vspace{10mm}


\textsf{\textsc{\noun{\LARGE University of Hradec Králové}}}\\
\vspace{0.5em}
\textsc{\noun{\LARGE Faculty of Informatics and Management}}\\
\vspace*{1em}
\textsf{\textsc{\noun{\Large Department of Information Technologies }}}

\vspace{15mm}

%\includegraphics[width=0.4\textwidth]{img/logo_uhk}

\vspace{15mm}


\textsf{\huge MASTER'S THESIS}{\huge \par}

\vspace{15mm}


\textsf{\LARGE Radio Fingerprint Acquisition Using a SmartWatch}{\LARGE \par}

\vspace{10mm}


\end{center}

\vspace*{\fill}


\vspace{10mm}


\begin{description}
\item [{{\large Author:}}] \noindent \textsf{\large Bc. David Sucharda}{\large \par}
\item [{{\large Study programme:}}] \noindent \textsf{\large Applied Informatics}{\large \par}
\item [{{\large Supervisor:}}] \noindent \textsf{\large Ing. Pavel Kříž, Ph.D.}

{\large \bigskip{}
}\noindent {\large{} Hradec Králové \hspace{\fill}April 2018}\\
{\large{}
}{\large \par}

\end{description}
\clearpage{}

%{\small \thispagestyle{plain}\addcontentsline{toc}{chapter}{Abstrakt} }{\small \par}

\newpage{}\thispagestyle{empty}

{\small %\setcounter{page}{3} % nastavení číslování stránek
\ }{\small \par}

\noindent {\small \vfill{}
 % nastavuje dynamické umístění následujícího textu do spodní části stránky
~}{\small \par}


% Declaration in Czech
\paragraph{Prohlášení}

\noindent {\small \\Prohlašuji, že jsem diplomovou práci vypracoval samostatně a uvedl jsem všechny použité prameny a literaturu.}{\small \par}
\vspace{5mm}

% Declaration in English
\paragraph{Declaration}

\noindent {\small \\I declare that I have elaborated this thesis independently and listed all the sources and literature.}{\small \par}
\vspace{25mm}

% Sign text
{\small \bigskip{}
}\noindent {\small{} Hradec Králové day 26th of April 2018\hspace{\fill}Bc. David Sucharda}\\
{\small{} % doplňte patřičné datum, jméno a příjmení
}{\small \par}

\clearpage{}

\newpage{}\thispagestyle{empty}

{\small %\setcounter{page}{3} % nastavení číslování stránek
\ }{\small \par}

\noindent {\small \vfill{}
 % nastavuje dynamické umístění následujícího textu do spodní části stránky
~}{\small \par}

\paragraph{Poděkování}

\noindent {\small \\Rád bych zde poděkoval Ing. Pavlu Kříži, Ph.D. za odborné vedení práce, podnětné rady a čas, který mi věnoval.}{\small \par}
\vspace{5mm}

\paragraph{Thanks}

\noindent {\small \\I would like to thank to Ing. Pavel Křiž, Ph.D. for professional guidance, incentive advices, and the time he gave me.\newpage{}}{\small \par}

\clearpage{}

\newpage{}\thispagestyle{empty}

{\small %\setcounter{page}{3} % nastavení číslování stránek
\ }{\small \par}

\noindent {\small \vfill{}
 % nastavuje dynamické umístění následujícího textu do spodní části stránky
~}{\small \par}

\paragraph{Anotace}
\noindent\textbf{\small \\Název práce: Sběr ráriových fingerprintů pomocí chytrých hodinek}{\small \par}
\noindent \\Text.

\paragraph{Annotation}
\noindent \\Text

\cleardoublepage{}

%\thispagestyle{empty}~{\small \addcontentsline{toc}{chapter}{Zadání
%práce} }{\small \par}

{\small %%%   Výtisk pak na tomto míste nezapomeňte PODEPSAT!
%%%                                         *********
}{\small \par}

\cleardoublepage{}\thispagestyle{empty}{\small
%\setcounter{secnumdepth}{3}
\setcounter{tocdepth}{3}%hloubla obsahu
\pagenumbering{gobble}
\tableofcontents{}% vkládá automaticky generovaný obsah dokumentu
\listoffigures{}
\listoftables{}

\thispagestyle{empty}

\chapter{Introduction}\label{sec:Introduction}
\pagenumbering{arabic}
\setcounter{page}{1}
As the technology evolves it unlocks more and more possibilities. Just few years back there were no such things as smart phones or smart watches but now they are important part of our lives. And as they evolve there is the need for them to have more functions. One of them is to locate it's position on the map. This is possible using Global Navigation Satellite System (GNSS). There is multiple implementation of this system like GPS, GLONASS or Galileo. All of these systems provide location using sufficient number (at least 4) of satellites. \iffalse[https://books.google.cz/books?id=Np7y43HU_m8C&printsec=frontcover&hl=cs#v=onepage&q&f=false]\fi GNSS  solution requires clear path between satellites and the device so It cannot be used indoor because the signal is not able to pass through buildings.
\\That is why there needs to be another solution for indoor localization. 

\section{Reason for selection of this theme}\label{sec:ReasonForSelectionOfThisTheme}


\begin{thebibliography}{10}

\bibitem{GNSS}
Bernhard Hofmann-Wellenhof, Herbert Lichtenegger and Elmar Wasle. \textit{GNSS – Global Navigation Satellite Systems: GPS, GLONASS, Galileo, and more}. Springer Science \& Business Media, 2007 [cited 2018-01-10], ISBN 9783211730171.

\bibitem{GNSSGPS}
AviationChief. \textit{Global Navigation Satellite System (GNSS) Global Positioning Satellite (GPS) System} [online]. AviationChief.Com, 2017 [cited 2018-01-15]. Available at: \url{http://www.aviationchief.com/gps-system.html}

\bibitem{LocalizationApproaches}
Xinglin Piao, Yong Zhang, Tingshu Li, Yongli Hu, Hao Liu, Ke Zhang and Yun Ge. \textit{RSS Fingerprint Based Indoor Localization Using Sparse Representation with Spatio-Temporal Constraint} [online]. National Center for Biotechnology Information, 2016 [cited 2018-01-14], Available at: \url{https://www.ncbi.nlm.nih.gov/pmc/articles/PMC5134504/}

\bibitem{PedestrianDeadReckoning}
Stéphane Beauregard and Harald Haas. \textit{Pedestrian Dead Reckoning: Basis for Personal Positioning} [online]. School of Engineering and Science
International University Bremen, 2006, Available at: \url{http://ave.dee.isep.ipp.pt/~lbf/PINSFUSION/BeHa06.pdf}

\bibitem{AaPLocalisation}
Gabriel Deak, Kevin Curran and Joan Condell. \textit{A survey of active and passive indoor localisation systems}. In: \textit{Computer Communications}. Elsevier, 2012 [cited 2018-01-11], Volume 35, Issue 16, ISSN: 0140-3664.

\bibitem{RAinWILTaS}
Zahid Farid, Rosdiadee Nordin, and Mahamod Ismail. \textit{Recent Advances in Wireless Indoor Localization Techniques and System} [online]. School of Electrical, Electronics \& System Engineering, University Kebangsaan Malaysia (UKM), 2013 [cited 2018-01-15], Available at: \url{http://downloads.hindawi.com/journals/jcnc/2013/185138.pdf}

\bibitem{LTinWSN}
Shweta Singh, Ravi Shakya and Yaduvir Singh. \textit{Localization techniques in wireless sensor networks} [online]. Department of Computer Science,
Ideal Institute of Technology, Ghaziabad, 2015 [cited 2018-01-15], ISSN: 0975-9646, Available at: \url{https://pdfs.semanticscholar.org/6299/85defbf9cc1a937a1b88c9c2a893552e3d89.pdf}

\bibitem{AoALforWSN}
Paweł Kułakowski, Javier Vales-Alonso, Esteban Egea-López, Wiesław Ludwin and Joan García-Haro. \textit{Angle-of-arrival localization based on antenna arrays for wireless sensor networks} [online]. In: \textit{Computers \& Electrical Engineering}. Elsevier, 2010 [cited 2018-01-15], Volume 36, Issue 6, Pages 1181-1186. Available at: \url{http://ai2-s2-pdfs.s3.amazonaws.com/17c6/0e17c4e72cc3fd821e12169c1c2ca7736bd4.pdf}

\bibitem{IILUBLEB}
Pavel Kriz, Filip Maly, and Tomas Kozel. \textit{Improving Indoor Localization Using Bluetooth Low Energy Beacons} [online]. In: \textit{Mobile Information Systems}. Hindawi Publishing Corporation, 2016 [cited 2018-01-15], Volume 2016, Article ID 2083094. Available at: \url{https://www.hindawi.com/journals/misy/2016/2083094/abs/}

\bibitem{TvTHGPSRW}
GISGeography. \textit{Trilateration vs Triangulation – How GPS Receivers Work} [online]. GISGeography.com, 2018 [cited 2018-01-15]. Available at: \url{http://gisgeography.com/trilateration-triangulation-gps/}

\bibitem{HPwAA}
Kenjirou Fujii, Yoshihiro Sakamoto, Wei Wang, Hiroaki Arie, Alexander Schmitz and Shigeki Sugano. \textit{Hyperbolic Positioning with Antenna Arrays and Multi-Channel Pseudolite for Indoor Localization} [online]. MDPI AG, Basel, 2015 [cited 2018-01-15]. Available at: \url{http://www.mdpi.com/1424-8220/15/10/25157/htm}

\bibitem{PLTaA}
David Munoz, Frantz Bouchereau Lara, Cesar Vargas and Rogerio Enriquez-Caldera. \textit{Position Location Techniques and Applications}. Elsevier Science Publishing Co Inc, 2009 [cited 2018-01-15], ISBN: 9780080921938. Available at: \url{http://www.mdpi.com/1424-8220/15/10/25157/htm}

\bibitem{AoA}
Group 891: Wireless Location. \textit{ANGULATION: AOA (Angle Of Arrival)} [online]. DEPARTMENT OF ELECTRONIC SYSTEMS, Aalborg University, 2010 [cited 2018-01-15]. Available at: \url{http://kom.aau.dk/group/10gr891/methods/Triangulation/Angulation/ANGULATION.pdf}

\bibitem{RofAoA}
Jais, M. I., Ehkan, P., Ahmad, R. B., Ismail, I., Sabapathy, T., and Jusoh, M. \textit{Review of angle of arrival (AOA) estimations through received signal strength indication (RSSI) for wireless sensors network (WSN)} [online]. In: Computer, Communications, and Control Technology (I4CT), 2015 International Conference on. IEEE, 2015, [cited 2018-01-16], p. 354-359. Available at: \url{https://www.researchgate.net/profile/Phaklen_Ehkan/publication/283476641_Review_of_angle_of_arrival_AOA_estimations_through_received_signal_strength_indication_RSSI_for_wireless_sensors_network_WSN/links/564106b008aebaaea1f6d6e5/Review-of-angle-of-arrival-AOA-estimations-through-received-signal-strength-indication-RSSI-for-wireless-sensors-network-WSN.pdf}

\bibitem{QAoA}
Quuppa Oy. \textit{Quuppa Intelligent Locating System} [online]. 2018 [cited 2018-01-16]. Available at: \url{http://quuppa.com/technology/}

\bibitem{ILWTP}
Krishna Chintalapudi, Anand Padmanabha Iyer, and Venkata N. Padmanabhan. \textit{Indoor Localization Without the Pain} [online]. In: Proceedings of the sixteenth annual international conference on Mobile computing and networking, 2010 [cited 2018-01-16], Available at: \url{http://dl.acm.org/citation.cfm?id=1860016}

\bibitem{RSSFofIFD}
Xiaoyang Wen, Wenyuan Tao, Chung-Ming Own, and Zhenjiang Pan. \textit{On the Dynamic RSS Feedbacks of Indoor Fingerprinting Databases for Localization Reliability Improvement} [online]. Sensors, 2016 [cited 2018-01-16], Available at: \url{https://www.ncbi.nlm.nih.gov/pmc/articles/PMC5017443/}

\bibitem{WiFiLBS}
Cisco. \textit{Wi-Fi Location-Based Services 4.1 Design Guide - Location Tracking Approaches} [online]. Cisco, 2018 [cited 2018-01-16], Available at: \url{https://www.cisco.com/c/en/us/td/docs/solutions/Enterprise/Mobility/WiFiLBS-DG/wifich2.html}

\bibitem{LSfUC}
Jeffrey Hightower and Gaetano Borriello. \textit{Location systems for ubiquitous computing} [online]. Computer, 2001 [cited 2018-01-17], 34.8: 57-66. Available at: \url{http://www.csd.uoc.gr/~hy439/lectures11/hightower2001survey.pdf}

\bibitem{LSAWIFI}
COOK, B., et al. \textit{Location by scene analysis of wi-fi characteristics} [online]. Relation, 2009 [cited 2018-01-17], 10.1.119: 6216. Available at: \url{http://www.ee.ucl.ac.uk/lcs/previous/LCS2006/2.pdf}

\bibitem{DRNS}
Levi, R.W. and Judd, T. \textit{Dead reckoning navigational system using accelerometer to measure foot impacts} [online]. Google Patents, 1996 [cited 2018-01-17]. Available at: \url{https://www.google.com/patents/US5583776}

\bibitem{IDRAIP}
Z. Zhou, T. Chen, L. Xu. \textit{An Improved Dead Reckoning Algorithm for Indoor Positioning Based on Inertial Sensors} [online]. In: Advances in Engineering Research, 2015 [cited 2018-01-17]. ISBN: 978-94-62520-71-4. Available at: \url{https://www.atlantis-press.com/proceedings/eame-15/22314}

\bibitem{IPBLEIUMWD}
NAKAJIMA, Naoki, et al. \textit{Improving Precision of BLE-based Indoor Positioning by Using Multiple Wearable Devices} [online]. In: Adjunct Proceedings of the 13th International Conference on Mobile and Ubiquitous Systems: Computing Networking and Services. ACM, 2016 [cited 2018-03-26]. p. 118-123. Available at: \url{https://dl.acm.org/citation.cfm?id=3004041}

\bibitem{SmartFix}
WANG, Xiaoliang; XU, Ke; LI, Ziwei. \textit{SmartFix: Indoor Locating Optimization Algorithm for Energy-Constrained Wearable Devices.} In: Wireless Communications and Mobile Computing, 2017 [cited 2018-03-26]. Available at: \url{https://dl.acm.org/citation.cfm?id=3004041}

\bibitem{TinyLoc}
W. Xiaoliang, X. Ke, Y. Zheng, and Z. Ge. \textit{Tinyloc: Indoor localization for energy-constrained wearable devices} [online]. In: Chinese Journal of Computers, 2016 (Chinese), [cited 2018-03-26]. Available at: \url{http://www.cnki.net/kcms/detail/11.1826.TP.20161106.1649.002.html}

\bibitem{MoLoc}
SUN, Wei, et al. \textit{MoLoc: On distinguishing fingerprint twins} [online]. In: Distributed Computing Systems (ICDCS), 2013 IEEE 33rd International Conference on. IEEE, 2013 [cited 2018-03-26]. p. 226-235. Available at: \url{http://ieeexplore.ieee.org/abstract/document/6681592/}

\bibitem{SWvsSP}
HOELZL, Gerold, et al. \textit{Size does matter-positioning on the wrist a comparative study: Smartwatch vs. smartphone} [online]. In: Pervasive Computing and Communications Workshops (PerCom Workshops), 2017 IEEE International Conference on. IEEE, 2017 [cited 2018-03-30]. p. 703-708.

\bibitem{WIGA}
Marziah Karch. \textit{What Is Google Android?} [online]. Lifewire, 2017 [cited 2018-01-17]. Available at: \url{https://www.lifewire.com/what-is-google-android-1616887}

\bibitem{AOSP}
Android. \textit{Android Open Source Code} [online]. Android.com, 2018 [cited 2018-01-17]. Available at: \url{https://source.android.com/}

\bibitem{AD}
Android. \textit{Android Developers} [online]. Android.com, 2018 [cited 2018-01-17]. Available at: \url{https://developer.android.com/index.html}

\bibitem{UtCoAWO}
Renju Liu and Felix Xiaozhu Lin. \textit{Understanding the Characteristics of Android Wear OS} [online]. In: Proceedings of the 14th Annual International Conference on Mobile Systems, Applications, and Services. ACM, 2016. p. 151-164. Available at: \url{https://athena.smu.edu.sg/mobisys/backend/mobisys/assets/paper_list/pdf_version/paper_12.pdf}

\bibitem{MIWD}
Samuel Gibbs. \textit{10 most influential wearable devices} [online]. Guardian News, 2017 [cited 2018-01-18]. Available at: \url{https://www.theguardian.com/technology/2017/mar/03/10-most-influential-wearable-devices}

\bibitem{GSWWDS}
Gartner, Inc. \textit{Gartner Says Worldwide Wearable Device Sales to Grow 17 Percent in 2017} [online]. Gartner, Inc., 2017 [cited 2018-01-18]. Available at: \url{https://www.gartner.com/newsroom/id/3790965}

\bibitem{SoASTaD}
Bahman Rashidi and Carol Fung. \textit{A Survey of Android Security Threats and Defenses} [online]. JoWUA, 2015, [cited 2018-01-19]. Available at: \url{https://www.researchgate.net/profile/Bahman_Rashidi2/publication/282365848_A_Survey_of_Android_Security_Threats_and_Defenses/links/560ec06908ae6b29b499a51f/A-Survey-of-Android-Security-Threats-and-Defenses.pdf}

\bibitem{ASIMPD}
Parvez Faruki, Ammar Bharmal, Vijay Laxmi, Vijay Ganmoor, Manoj Singh Gaur and Mauro Conti. \textit{Android Security: A Survey of Issues, Malware Penetration and Defenses} [online]. IEEE Communications Surveys and Tutorials, 17(2), pp. 998-1022, 2015, [cited 2018-01-19]. Available at: \url{http://dx.doi.org/10.1109/COMST.2014.2386139}

\bibitem{NoAAiGPS}
Statista. \textit{Number of available applications in the Google Play Store from December 2009 to December 2017} [online]. Statista, 2018, [cited 2018-01-19]. Available at: \url{https://www.statista.com/statistics/266210/number-of-available-applications-in-the-google-play-store/}

\bibitem{NoAA}
AppBrain. \textit{Number of Android applications} [online]. AppBrain, 2018, [cited 2018-01-19]. Available at: \url{https://www.appbrain.com/stats/number-of-android-apps}

\bibitem{CSUITW}
LIU, Xing, et al. \textit{Characterizing Smartwatch Usage In The Wild} [online]. In: Proceedings of the 15th Annual International Conference on Mobile Systems, Applications, and Services. ACM, 2017 [cited 2018-01-19]. p. 385-398. Available at: \url{https://pdfs.semanticscholar.org/0cc2/4bccc3067ed688e576603bc6bab0e5e1b1db.pdf}

\bibitem{UCAW}
Liu, Renju, and Felix Xiaozhu Lin. \textit{Understanding the Characteristics of Android Wear OS} [online]. In: Proceedings of the 14th Annual International Conference on Mobile Systems, Applications, and Services. ACM, 2016 [cited 2018-01-19], p. 151-164. Available at: \url{https://athena.smu.edu.sg/mobisys/backend/mobisys/assets/paper_list/pdf_version/paper_12.pdf}

\bibitem{AWPaS}
Sandra Henshaw. \textit{Android Wear Problems (and Solutions!)} [online]. Tiger Mobiles Limited, 2016 [cited 2018-01-20]. Available at: \url{https://www.tigermobiles.com/2016/01/android-wear-problems-and-solutions/}

\bibitem{WAWP}
Simon Hill. \textit{10 of the worst Android Wear problems, and how to fix them} [online]. Designtechnica Corporation, 2017 [cited 2018-01-20]. Available at: \url{https://www.digitaltrends.com/wearables/android-wear-problems/}

\bibitem{TOAW}
ADNAN F. \textit{Tizen overtakes Android Wear in smartwatch market share} [online]. SamMobile, 2017 [cited 2018-01-20]. Available at: \url{https://www.sammobile.com/2017/05/11/tizen-overtakes-android-wear-in-smartwatch-market-share/}

\bibitem{AW2UG}
Paul Lamkin. \textit{Android Wear 2.0: Ultimate guide to the major smartwatch update} [online]. Wareable, 2017 [cited 2018-01-20]. Available at: \url{https://www.wareable.com/android-wear/android-wear-update-everything-you-need-to-know-2735}

\bibitem{AW2WN}
Elyse Betters and Chris Hall. \textit{Android Wear 2.0: What's new in the major software update for watches?} [online]. Pocket-lint Limited, 2017 [cited 2018-01-20]. Available at: \url{https://www.pocket-lint.com/smartwatches/news/google/139007-android-wear-2-0-what-s-new-in-the-major-software-update-for-watches}

\bibitem{AW2N}
Chris Martin. \textit{Android Wear 2.0 news: release date and features} [online]. Tech Advisor, 2017 [cited 2018-01-20]. Available at: \url{https://www.techadvisor.co.uk/new-product/google-android/android-wear-2-3640616/}

\bibitem{DoAW}
Android Developers. \textit{Designing for Android Wear} [online]. Android, 2018 [cited 2018-01-20]. Available at: \url{https://developer.android.com/design/wear/index.html}

\bibitem{WIGA}
Elyse Betters. \textit{What is Google Assistant, how does it work, and which devices offer it?} [online]. Pocket-lint Limited, 2018 [cited 2018-01-20]. Available at: \url{https://www.pocket-lint.com/apps/news/google/137722-what-is-google-assistant-how-does-it-work-and-which-devices-offer-it}

\bibitem{ASGA}
DAN MOREN. \textit{Alexa vs. Siri vs. Google Assistant: Which Smart Assistant Wins?} [online]. Tom's Guide, 2017 [cited 2018-01-20]. Available at: \url{https://www.tomsguide.com/us/alexa-vs-siri-vs-google,review-4772.html}

\bibitem{VACCGASAB}
Digital Trends Staff. \textit{Virtual assistant comparison: Cortana, Google Assistant, Siri, Alexa, Bixby} [online]. Digital Trends, 2017 [cited 2018-01-20]. Available at: \url{https://www.digitaltrends.com/computing/cortana-vs-siri-vs-google-now/}

\bibitem{CAGACS}
Brian Heater. \textit{Comparing Alexa, Google Assistant, Cortana and Siri smart speakers} [online]. TechCrunch, 2017 [cited 2018-01-20]. Available at: \url{https://techcrunch.com/2017/10/08/comparing-alexa-google-assistant-cortana-and-siri-smart-speakers/}

\bibitem{GASBAC}
Joe Hindy. \textit{Google Assistant vs Siri vs Bixby vs Amazon Alexa vs Cortana – Best virtual assistant showdown!} [online]. Android Authority, 2017 [cited 2018-01-20]. Available at: \url{https://www.androidauthority.com/google-assistant-vs-siri-vs-bixby-vs-amazon-alexa-vs-cortana-best-virtual-assistant-showdown-796205/}

\bibitem{XRN4FPS}
GSMArena. \textit{Xiaomi Redmi Note 4 - Full phone specifications} [online]. GSMArena, 2018 [cited 2018-01-21]. Available at: \url{https://www.gsmarena.com/xiaomi_redmi_note_4-8531.php}

\bibitem{XRN4LTE}
XiaomiMobile. \textit{Xiaomi Redmi Note 4 LTE} [online]. XiaomiMobile, 2018 [cited 2018-01-21]. Available at: \url{https://xiaomimobile.cz/xiaomi-redmi-note-4-pro-lte-global.html?search_query=note+4&results=16}

\bibitem{BAWW}
Android Authority Team. \textit{Best Android Wear watches (old version)} [online]. Android Authority, 2017 [cited 2018-01-21]. Available at: \url{https://www.androidauthority.com/best-android-watches-572773/}

\bibitem{BAWW18}
James Peckham. \textit{Best Android Wear watch 2018: our list of the top Google OS smartwatches} [online]. TechRadar, 2017 [cited 2018-01-21]. Available at: \url{http://www.techradar.com/news/wearables/every-android-wear-smartwatch-in-the-world-today-1288283}

\bibitem{BAWW17}
Michael Simon. \textit{Best Android Wear watches of 2017} [online]. PCWorld, 2017 [cited 2018-01-21]. Available at: \url{https://www.pcworld.com/article/3209668/android/best-android-wear-watches-of-2017.html}

\bibitem{LGWSP}
LG Electronics. \textit{LG Watch Sport™ - AT\&T} [online]. LG Electronics, 2018 [cited 2018-01-22]. Available at: \url{http://www.lg.com/us/smart-watches/lg-W280A-sport}

\bibitem{LGWST}
LG Electronics. \textit{LG Watch Style} [online]. LG Electronics, 2018 [cited 2018-01-22]. Available at: \url{http://www.lg.com/us/smart-watches/lg-W270-Titanium-style}

\bibitem{HW2}
HUAWEI Technologies Co. \textit{HUAWEI WATCH 2} [online]. HUAWEI Technologies Co., 2018 [cited 2018-01-22]. Available at: \url{http://consumer.huawei.com/en/wearables/watch2/specs/}

\bibitem{PM600}
Polar Electro. \textit{Polar M600} [online]. Polar Electro, 2018 [cited 2018-01-22]. Available at: \url{https://support.polar.com/e_manuals/M600/Polar_M600_user_manual_English/Content/technical-specifications.htm}

\bibitem{AZW3}
ASUSTeK Computer Inc. \textit{ASUS ZenWatch 3} [online]. ASUSTeK Computer Inc., 2018 [cited 2018-01-22]. Available at: \url{https://www.asus.com/us/ZenWatch/ASUS-ZenWatch-3-WI503Q/specifications/}

\bibitem{HtPAWW}
Adam Conway. \textit{How to Pair Android Wear Watches to New Phones without Factory Resetting} [online]. xda-developers, 2017 [cited 2018-01-22]. Available at: \url{https://www.xda-developers.com/pair-android-wear-without-factory-reset/}

\bibitem{AWITFNM}
Dennis Troper. \textit{Android Wear, it’s time for a new name} [online]. Google, 2018 [cited 2018-04-02]. Available at: \url{https://www.blog.google/products/wear-os/android-wear-its-time-new-name/}

\bibitem{IPSBOBLE}
Locatify. \textit{Indoor Positioning Systems based on BLE Beacons – Basics} [online]. Locatify, 2015 [cited 2018-01-27]. Available at: \url{https://locatify.com/blog/indoor-positioning-systems-ble-beacons/}

\bibitem{10TABB}
Patrick Leddy. \textit{10 Things About Bluetooth Beacons You Need to Know} [online]. pulsate, 2015 [cited 2018-01-27]. Available at: \url{http://academy.pulsatehq.com/bluetooth-beacons}

\bibitem{RMPFEB}
The Estimote Team Blog. \textit{Reality matters — Preorder for Estimote Beacons available, shipping this summer} [online]. Estimote, Inc., 2013 [cited 2018-01-27]. Available at: \url{http://blog.estimote.com/post/57087851702/preorder-for-estimote-beacons-available-shipping}

\bibitem{ESDKfA}
\textit{Estimote SDK for Android} [online]. Estimote, 2018 [cited 2018-01-22]. Available at: \url{https://github.com/Estimote/Android-SDK}

\bibitem{PMRIL}
Radek Brůha. \textit{Pokročilé metody rádiové indoor lokalizace} [online]. Univerzita Hradec Králové, 2017 [cited 2018-01-22]. Available at: \url{https://theses.cz/id/jss047}

\bibitem{ABL}
\textit{Android Beacon Library} [online]. Radius Networks, 2018 [cited 2018-01-22]. Available at: \url{https://altbeacon.github.io/android-beacon-library/}

\bibitem{AltB}
\textit{AltBeacon} [online]. AltBeacon, 2018 [cited 2018-01-22]. Available at: \url{http://altbeacon.org/}

\bibitem{EDDF}
\textit{Eddystone format} [online]. Google Developers, 2018 [cited 2018-01-22]. Available at: \url{https://developers.google.com/beacons/eddystone}

\bibitem{NOSQLDB}
\textit{NOSQL Databases} [online]. NoSQL, 2018 [cited 2018-01-27]. Available at: \url{http://nosql-database.org/}

\bibitem{NOSQLDB}
\textit{NOSQL Databases} [online]. NoSQL, 2018 [cited 2018-01-27]. Available at: \url{http://nosql-database.org/}

\bibitem{GSWCBS}
Brown, Martin C. \textit{Getting Started with Couchbase Server: Extreme Scalability at Your Fingertips} [online]. O'Reilly Media, Inc., 2012 [cited 2018-01-27]. Available at: \url{https://books.google.cz/books?hl=cs&lr=&id=5xu33G9LGkMC&oi=fnd&pg=PR5&dq=Couchbase&ots=nrw7O3HiVh&sig=6qmpVJLxwdOK9RRZsICHUfsD2wI&redir_esc=y#v=onepage&q&f=false}

\bibitem{WINQL}
\textit{What is N1QL?} [online]. Couchbase, 2018 [cited 2018-01-27]. Available at: \url{https://www.couchbase.com/products/n1ql}

\bibitem{ERDMS}
\textit{Explain Relational Database Management System (RDBMS)} [online]. W3Schools, 2016 [cited 2018-01-23]. Available at: \url{http://whatisdbms.com/explain-relational-database-management-system-rdbms/}

\bibitem{WISQLITE}
\textit{What Is SQLite} [online]. SQLite Tutorial, 2018 [cited 2018-01-23]. Available at: \url{http://www.sqlitetutorial.net/what-is-sqlite/}

\bibitem{WTM}
Sterling Quinn, John A. Dutto. \textit{Why tiled maps?} [online]. e-Education Institute, College of Earth and Mineral Sciences, The Pennsylvania State University, 2018 [cited 2018-04-08]. Available at: \url{https://www.e-education.psu.edu/geog585/node/706}

\bibitem{TileView}
Mike Dunn. \textit{TileView} [online]. Mike Dunn, 2016 [cited 2018-04-11]. Available at: \url{https://github.com/moagrius/TileView}

\bibitem{SOTAJS}
\textit{Scheduling of tasks with the Android JobScheduler - Tutorial} [online]. vogella, 2017 [cited 2018-04-06]. Available at: \url{http://www.vogella.com/tutorials/AndroidTaskScheduling/article.html}

\bibitem{GTMDL}
\textit{Google Tag Manager DataLayer Explained} [online]. Analytics Mania, 2017 [cited 2018-04-08]. Available at: \url{https://www.analyticsmania.com/post/what-is-data-layer-in-google-tag-manager/}

\end{thebibliography}


\addcontentsline{toc}{chapter}{Literature}
\cleardoublepage{}

%\include{prilohy}

% Include pdf scan of Thesis entry
%\includepdf[pages={1}]{zadani.pdf}

\end{document}