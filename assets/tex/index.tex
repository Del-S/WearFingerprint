%% LyX 1.5.5 created this file.  For more info, see http://www.lyx.org/.
%% Do not edit unless you really know what you are doing.
\documentclass[a4paper,english,english,openright,cleardoubleempty,BCOR10mm,DIV11,12pt]{scrreprt}
\usepackage[T1]{fontenc}
\usepackage[utf8]{inputenc}
\usepackage[english]{babel}
\usepackage[table]{xcolor} 
\usepackage{array}
\usepackage{float}
%\usepackage{longtable}
\usepackage{varioref}
\usepackage{wrapfig}
\usepackage{fancybox}
\usepackage{calc}
\usepackage{framed}
\usepackage{url}
\def\UrlBreaks{\do\/\do-}
\usepackage{graphicx}
\usepackage{placeins} %floatbarrier \FloatBarrier
%\usepackage{listing}
\usepackage{pdfpages}
\usepackage{epstopdf}
\usepackage[left=3cm,top=2.5cm,right=2.5cm,bottom=2.5cm]{geometry}
\usepackage{breakurl}
\usepackage{indentfirst}
\usepackage[nottoc,notlot,notlof]{tocbibind}
\usepackage{refcount}
\usepackage{xstring} 
\usepackage{enumerate}
\usepackage{csquotes}


% Cut words in half
%\hyphenation{Word-cut}

\makeatletter

\usepackage[font=small,labelfont=bf]{caption} %captiony

% Enable break url with "_" (underscore)
\expandafter\def\expandafter\UrlBreaks\expandafter{\UrlBreaks%  save the current one
	\do\_}

%%%%%%%%%%%%%%%%%%%%%%%%%%%%%% LyX specific LaTeX commands.
\providecommand{\LyX}{L\kern-.1667em\lower.25em\hbox{Y}\kern-.125emX\@}
\newcommand{\lyxline}[1][1pt]{%
  \par\noindent%
  \rule[.5ex]{\linewidth}{#1}\par}
\newcommand{\noun}[1]{\textsc{#1}}
%% Special footnote code from the package 'stblftnt.sty'
%% Author: Robin Fairbairns -- Last revised Dec 13 1996
\let\SF@@footnote\footnote
\def\footnote{\ifx\protect\@typeset@protect
    \expandafter\SF@@footnote
  \else
    \expandafter\SF@gobble@opt
  \fi
}

\renewcommand{\baselinestretch}{1.5} %line height

\expandafter\def\csname SF@gobble@opt \endcsname{\@ifnextchar[%]
  \SF@gobble@twobracket
  \@gobble
}
\edef\SF@gobble@opt{\noexpand\protect
  \expandafter\noexpand\csname SF@gobble@opt \endcsname}
\def\SF@gobble@twobracket[#1]#2{}
%% Because html converters don't know tabularnewline
\providecommand{\tabularnewline}{\\}

%%%%%%%%%%%%%%%%%%%%%%%%%%%%%% Textclass specific LaTeX commands.
\newenvironment{lyxcode}
{\begin{list}{}{
\setlength{\rightmargin}{\leftmargin}
\setlength{\listparindent}{0pt}% needed for AMS classes
\raggedright
\setlength{\itemsep}{0pt}
\setlength{\parsep}{0pt}
\normalfont\ttfamily}%
 \item[]}
{\end{list}}

%%%%%%%%%%%%%%%%%%%%%%%%%%%%%% User specified LaTeX commands.
%<-------------------------------společná nastavení------------------------------>
%s\usepackage[]{babel}%počeštění názvů (Obsah, Kapitola, Literatura atp.)
\usepackage[]{hyperref} %odkazy v  pdf jsou klikací s barevnými rámečky
\usepackage[numbers,sort&compress]{natbib} %balíček pro citace literatury
%\usepackage{hypernat}%interakce mezi hyperref a natbib
%\newcommand{\BibTeX}{{\sc Bib}\TeX}%BibTeX logo
\hypersetup{   % Nastavení polí PDF dokumentu
pdftitle={Radio Fingerprint Acquisition Using Smartwatch},%
pdfauthor={Bc. David Sucharda},%
pdfsubject={},%
pdfkeywords={Fingerprint, Android, Wear, RSS}%
}
\usepackage{multicol}

%<-----------------------------volání stylů----------------------------------------->
%<------------------------------------písmo----------------------------------------->
%\usepackage{pkg/bc-latinmodern}
%\usepackage{pkg/bc-times}
\usepackage{pkg/bc-palatino}
%\usepackage{pkg/bc-iwona}
%\usepackage{pkg/bc-helvetika}


%<------------------------------záhlaví stránek------------------------------------>
%\usepackage{pkg/bc-headings}
\usepackage{pkg/bc-fancyhdr}

%<------------------------------hlavičky kapitol------------------------------------>
%\usepackage{pkg/bc-neueskapitel}
%\usepackage{pkg/bc-fancychap}

\makeatother

\usepackage{babel}

%java code block%

\usepackage{listing}
\usepackage{listings}
\usepackage{color}

\definecolor{dkgreen}{rgb}{0,0.6,0}
\definecolor{gray}{rgb}{0.5,0.5,0.5}
\definecolor{mauve}{rgb}{0.58,0,0.82}

\renewcommand*{\lstlistingname}{Code example} %prejmenovani
\renewcommand*{\lstlistlistingname}{List of code examples}

% syntax highlight pro jazyk Java %
\lstset{
  %frame=r,
  captionpos=b,
  language=Java,
  aboveskip=3mm,
  belowskip=3mm,
  xleftmargin=0.2mm,
  showstringspaces=false,
  columns=flexible,
  basicstyle={\small\ttfamily},
  numbers=none,
  numberstyle=\tiny\color{gray},
  keywordstyle=\color{blue},
  commentstyle=\color{dkgreen},
  stringstyle=\color{mauve},
  breaklines=true,
  breakatwhitespace=true,
  tabsize=3,
    inputencoding=utf8,
    extendedchars=true,
    literate=%
    {á}{{\'a}}1
    {č}{{\v{c}}}1
    {ď}{{\v{d}}}1
    {é}{{\'e}}1
    {ě}{{\v{e}}}1
    {í}{{\'i}}1
    {ň}{{\v{n}}}1
    {ó}{{\'o}}1
    {ř}{{\v{r}}}1
    {š}{{\v{s}}}1
    {ť}{{\v{t}}}1
    {ú}{{\'u}}1
    {ů}{{\r{u}}}1
    {ý}{{\'y}}1
    {ž}{{\v{z}}}1
    {Á}{{\'A}}1
    {Č}{{\v{C}}}1
    {Ď}{{\v{D}}}1
    {É}{{\'E}}1
    {Ě}{{\v{E}}}1
    {Í}{{\'I}}1
    {Ň}{{\v{N}}}1
    {Ó}{{\'O}}1
    {Ř}{{\v{R}}}1
    {Š}{{\v{S}}}1
    {Ť}{{\v{T}}}1
    {Ú}{{\'U}}1
    {Ů}{{\r{U}}}1
    {Ý}{{\'Y}}1
    {Ž}{{\v{Z}}}1
}

\begin{document}
\renewcommand{\figurename}{Figure}
\renewcommand{\tablename}{Table}
\renewcommand{\contentsname}{Content}
\renewcommand{\bibname}{Literature}
\renewcommand{\listfigurename}{List of figures}
\renewcommand{\listtablename}{List of tables}
%~\thispagestyle{empty}{\small ~\vfill{}
%}{\small \par}

% Add command (\fref) to reference Figures
\newcommand\fref[1]{\IfRefUndefinedExpandable{#1}{??}{\figurename~\ref{#1}}}
% Add command (\fref) to reference Tables
\newcommand\tref[1]{\IfRefUndefinedExpandable{#1}{??}{\tablename~\ref{#1}}}

%~\thispagestyle{empty}\vfill{}
%Tato stránka je tzv. protititul a je graficky součástí titulní stránky.
%Nechte ji prázdnou, nebo na ni umístěte vhodnou fotografii či ilustraci.

\cleardoublepage{}~\thispagestyle{empty}\begin{center}\pagenumbering{roman}\vspace{10mm}


\textsf{\textsc{\noun{\LARGE University of Hradec Králové}}}\\
\vspace{0.5em}
\textsc{\noun{\LARGE Faculty of Informatics and Management}}\\
\vspace*{1em}
\textsf{\textsc{\noun{\Large Department of Information Technologies }}}

\vspace{15mm}

%\includegraphics[width=0.4\textwidth]{img/logo_uhk}

\vspace{15mm}


\textsf{\huge MASTER'S THESIS}{\huge \par}

\vspace{15mm}


\textsf{\LARGE Radio Fingerprint Acquisition Using\\Smartwatch}{\LARGE \par}

\vspace{10mm}


\end{center}

\vspace*{\fill}


\vspace{10mm}


\begin{description}
\item [{{\large Author:}}] \noindent \textsf{\large Bc. David Sucharda}{\large \par}
\item [{{\large Study programme:}}] \noindent \textsf{\large Applied Informatics}{\large \par}
\item [{{\large Supervisor:}}] \noindent \textsf{\large Ing. Pavel Kříž, Ph.D.}

{\large \bigskip{}
}\noindent {\large{}Hradec Králové \hspace{\fill}April 2018}\\
{\large{}
}{\large \par}

\end{description}
\clearpage{}

\newpage{}\thispagestyle{empty}

{\small %\setcounter{page}{3} % nastavení číslování stránek
\ }{\small \par}

\noindent {\small \vfill{}
 % nastavuje dynamické umístění následujícího textu do spodní části stránky
~}{\small \par}


% Declaration in Czech
\paragraph{Prohlášení}

\noindent {\small \\Prohlašuji, že jsem diplomovou práci vypracoval samostatně a uvedl jsem všechny použité prameny a literaturu.}{\small \par}
\vspace{5mm}

% Declaration in English
\paragraph{Declaration}

\noindent {\small \\I declare that I have elaborated this thesis independently and listed all the sources and literature.}{\small \par}
\vspace{25mm}

% Sign text
{\small \bigskip{}
}\noindent {\small{}Hradec Králové day 26th of April 2018\hspace{\fill}Bc. David Sucharda}\\
{\small{} % doplňte patřičné datum, jméno a příjmení
}{\small \par}

\clearpage{}

\newpage{}\thispagestyle{empty}

{\small %\setcounter{page}{3} % nastavení číslování stránek
\ }{\small \par}

\noindent {\small \vfill{}
 % nastavuje dynamické umístění následujícího textu do spodní části stránky
~}{\small \par}

\paragraph{Poděkování}

\noindent {\small \\Rád bych zde poděkoval Ing. Pavlu Kříži, Ph.D. za odborné vedení práce, podnětné rady a čas, který mi věnoval.}{\small \par}
\vspace{5mm}

\paragraph{Thanks}

\noindent {\small \\I would like to thank to Ing. Pavel Křiž, Ph.D. for professional guidance, incentive advices, and the time he gave me.\newpage{}}{\small \par}

\clearpage{}

\newpage{}\thispagestyle{empty}

{\small %\setcounter{page}{3} % nastavení číslování stránek
\ }{\small \par}

\noindent {\small \vfill{}
 % nastavuje dynamické umístění následujícího textu do spodní části stránky
~}{\small \par}

\paragraph{Anotace}
\noindent\textbf{\small \\Název práce: Sběr rádiových fingerprintů pomocí chytrých hodinek}{\small \par}
\noindent \\Diplomová práce se zabývá možnostmi sběru rádiových otisků (fingerprintů) za pomoci chytrých hodinek. Tyto otisky se používají k lokalizaci uvnitř budovy. Hlavním cílem této práce je prozkoumat možnosti sběru otisků a návrh aplikace která bude tento sběr umožňovat. V první části práce je potřeba zjistit, jestli je tento sběr na hodinkách vůbec možný. V další části je zpracování aplikace na mobil a hodinky. A jako poslední část této práce je sběr otisků a jejich analýza. Jeden z osobních cílů je zpracovat tuto aplikaci aby byla co nejvíce uživatelky přívětivá.

\paragraph{Annotation}
\noindent \\The Master's thesis deals with possibilities of collecting radio fingerprints with the help of smart watches. These prints are used in indoor localization. Main aim of this thesis is to explore possibilities of fingerprint collection and creation of application that will allow it. First part is to figure out if this collection is even possible using smart watch. Next part deals with creation of such application not only for watch but also for the phone. And at the end part there is testing of fingerprint collection and data analysis. One of the personal goal is to make this application as user friendly as possible.

\cleardoublepage{}

\cleardoublepage{}\thispagestyle{empty}{\small
\setcounter{secnumdepth}{3}
\setcounter{tocdepth}{3}%hloubla obsahu
\pagenumbering{gobble}
\tableofcontents{}% vkládá automaticky generovaný obsah dokumentu
\listoffigures{}
\listoftables{}

\thispagestyle{empty}

\chapter{Introduction}\label{sec:Introduction}
\pagenumbering{arabic}
\setcounter{page}{1}
As the technology evolves it unlocks more and more possibilities. Just a few years back there were no smartwatches or phones but at this time they are important part of our lives. As they evolve there is the need for them to have more functions and features. One of these features is to be able to locate its position on the map. This information is very useful since it can prevent people from getting lost, figuring out path to drive, used by military and in countless more cases.

Finding such position is possible using Global Navigation Satellite System (GNSS). Multiple implementations of this system exist, such as GPS, GLONASS or Galileo. All of these systems provide location using sufficient number (at least four) of satellites \cite{GNSS, GNSSGPS}. GNSS solution requires clear line of sight between satellites and the receiving device because signal is not able to pass through buildings. This makes it the main reason why it cannot be used for indoor localization.

\begin{figure}[h!]
	\begin{centering}
		\includegraphics[width=0.7\textwidth]{img/1_comparison_of_positionin_technologies}
		\par\end{centering}
	\caption{Comparison of Positioning Technologies (source: \cite{PedestrianDeadReckoning})\label{fig:1_comparison_of_positionin_technologies}}
	\label{fig01c01}
\end{figure}

There are multiple approaches to find out location inside the building. They can be divided into three main types. First, using wireless signal ranging approach with multiple kinds of data such as Time of Arrival (ToA). Second, using special equipment like active bats (Ultrasonic). Final, based on Signal Strength Fingerprint Maps (SSFM), in which first part is to collect signal strengths from the environment and construct fingerprint maps. These maps are then used to match with current signal to obtain device location \cite{LocalizationApproaches}.

In addition to these types of localization there are also multiple algorithms used in indoor environments. Some of them are location fingerprinting, triangulation, proximity and dead reckoning \cite{AaPLocalisation}. Few of these algorithms will be described in \hyperref[sec:LocalizationTechniques]{Chapter 2}.

This thesis is focused on method using radio signal strength (RSS) fingerprinting by collecting data from Bluetooth, wireless and cellular devices.

\section{Goals of this thesis}\label{sec:GoalsOfThisThesis}
Main goal of this thesis is to explore possibilities of fingerprint acquisition using smartwatch technology. The first question that needs to be answered is if this can be done. Is smartwatch capable of RSS data collection? And the answer to this question is yes, since smartwatches have the similar specifications as low-end smartphones, containing Bluetooth and WiFi chips, which means it can be done. 

One of the goals for this thesis is to create an application for Android phone and wear device which handles RSS fingerprint collection. Problem with smartwatches is their diversity in operational systems because a lot of watch creators have their own custom systems which can complicate things. Luckily there is a version of Android operating system called WearOS (used to be Wear 2.0) and it is basically a port of Android system to wearable devices. 

Final goal is to test created application and figure out if data from wear device increase precision of indoor localization or not.

\section{Reason for selection of this topic}\label{sec:ReasonForSelectionOfThisTopic}
The reason behind selection of this topic is rather simple. I was introduced to Android during my studies at the University Hradec Králové but it was just basic knowledge. That is why I later decided to go for a study abroad to deepen my knowledge. Part of that study was to work for a selected company where we developed rather complex Android application. Its core part was using multiple APIs, user authentication and data encryption but it was still focused only on a singe device, that being the phone. So next thing I wanted to try was working with multiple kinds of devices and since WearOS is rather new I wanted to test it out. So the main reason is to get more experienced with Android and as a developer.
\chapter{Indoor localization using RSS Fingerprints}\label{sec:Indoor localization using RSS Fingerptints}
Text

\section{How does it work}
\label{sec:HowDoesItWork}

\section{Localization methods}
\label{sec:LocalizationMethods}
\chapter{Related Work}\label{sec:RelatedWork}
This is not a novel idea and there are already some completed solutions and papers written about RSS fingerprint collection using multiple devices. This chapter will describe few selected solutions close to this one as a comparison.

\section{Improving Precision by Using Multiple Wearable Devices}\label{sec:IPUMWD}
This paper focuses on improving indoor localization using BLE-based fingerprinting with multiple devices \cite{IPBLEIUMWD}. Using combination of smartphone and wear should in this case prevent signal obstruction from human body and at least one of the devices should receive beacon signal. Unfortunately due to low BLE sensibility of wear devices, authors decided to supplement them with a second mobile device, in this case with Nexus 5 running Android version 4.4.

It proposes calculating medians from 800 millisecond tests where user can move only one meter from starting position at most. Average and variance is calculated for all medians with the same position, based on these values a normal distribution is used to model the potential variation of RSSI. Important thing to note is that fingerprint maps are also built based on facing direction. These maps are then tested using five main scenarios.

\begin{itemize}
	\item P1: single device held in hand where body does not obstruct its line of sight (LOS) path to all beacons.
	\item P2: one device is placed in breast pocket where LOS may be obstructed to some beacons.
	\item P3: user holds smartphone in hand and wears a smartwatch on one wrist.
	\item P4: single device is placed in the breast pocket and the other is on one wrist.
	\item P5: one device is in the breast pocket, and two other devices, each on one wrist.
\end{itemize} 

At first, four of these scenarios were tested in a 15x8 meters entrance hall with four deployed beacons in the corners and ten measurement positions. Using multiple devices in this location improved position precision and reduced error by 57\%. \tref{tab01c03} shows mean errors for previously mentioned cases in this location where using more devices improves localization. However there is one position where case P4 will result with higher error than P3 due to building's structure and signal obstruction for nearest beacons.

\vspace*{6pt}
\begin{table}[h]
	\begin{center}
			\begin{tabular}{ l | C{3cm} }
				\hline
				Scenario & Mean error (m) \\ \hline
				P1 & 2.36 \\
				P2 & 1.71 \\
				P3 & 0.96 \\
				P4 & 0.41 \\ \hline
		\end{tabular}
		\caption{Mean errors at first location (sources: \cite{IPBLEIUMWD})}
		\label{tab01c03}
	\end{center}
\end{table} 
\vspace*{-\baselineskip}
\vspace*{6pt}

Second case, all of previously mentioned scenarios were tested in a conference room with unified ceiling and desks near the walls. This location is used to investigate the impact of beacon density to position precision. Three following combinations of beacons are used

\begin{itemize}
	\item A: four beacons at the corners,
	\item B: combination A with one beacon at the center,
	\item C: combination B with four more beacons at the sides of the room.
\end{itemize}

\vspace*{6pt}
\begin{table}[h]
	\begin{center}
		\begin{tabular}{ l | C{1.25cm} C{1.25cm} C{1.25cm} }
			\hline
			& \multicolumn{3}{ c }{Mean error (m)} \\ \hline
			Scenario & A & B & C \\ \hline
			P1 & 2.05 & 2.05 & 1.76 \\ 
			P2 & 1.84 & 1.49 & 0.90 \\ 
			P3 & 1.36 & 1.09 & 0.63 \\ 
			P4 & 1.24 & 0.78 & 0.23 \\ 
			P5 & 0.80 & 0.39 & 0.07 \\ \hline
		\end{tabular}
		\caption{Mean errors at second location (sources: \cite{IPBLEIUMWD})}
		\label{tab02c03}
	\end{center}
\end{table} 
\vspace*{-\baselineskip}
\vspace*{6pt}

Using directional maps at this location resulted in increase of maximum localization error, which was not expected. This error can increase even more when using higher count of beacons and occurs mostly when testing at the edges of the room. Mean error on the other hand shows an improvement, higher with more beacons used. 

In summary, position error can be improved by three aspects: using more devices, using directional map or increasing the number of beacons. There are two main conclusions of this paper. First, confirmed precision degradation when testing near the edges of the room with obstructed signal to nearest beacons. Second, human body does not greatly change radio signal and device can receive reflected signals with sufficient strength. 

\section{SmartFix}\label{sec:SmartFix}
Complete name of this paper is \enquote{An Indoor Locating Optimization Algorithm for Energy-Constrained Wearable Devices called SmartFix} \cite{SmartFix}. The main goal was set to improve energy consumption efficiency for wearable-based indoor localization systems using WiFi fingerprinting. In the begging, single real-time experiment of energy consumption was run and split into two main parts: computation of location and collection of fingerprint. According to this experiment, energy consumption for data collection is 99\% of localization algorithm.

This paper proposes novel indoor localization strategy, SmartFix, that can cooperate with an existing indoor localization technologies based on WiFi to decrease power consumption. It enhances accuracy of such algorithm with a little extra energy cost of calculation but large decrease of power consumption for signal collection. Aided with machine-learning algorithm, it obtains the relative features given by the trajectories of users in certain areas and modify the positioning results. SmartFix can save up to 70\% of energy while achieving the same localization accuracy when compared to the original fingerprinting method.

To test this new system it was implemented with prototypes of TinyLoc \cite{TinyLoc}, MoLoc \cite{MoLoc} and basic WiFi fingerprinting method using K-Nearest Neighbors algorithm. TinyLoc is more focused on energy efficiency than location accuracy. In contrast, SmartFix analyses the history of people trajectory in given area to improve localization results by referring to user motion features. SmartFix then modifies positional results to achieve satisfying accuracy. MoLoc, same as SmartFix, also leverages user motion by collecting trajectory patterns using device built-in sensors to improve localization.

All previously mentioned, prototypes were deployed with and without SmartFix to test their power consumptions. This algorithm only needs a single real-time RSS signal in the locating phase to guarantee an excellent energy saving performance. \fref{fig01c03} shows power consumption of on-time locating on two specific devices: HTC one and Moto 360. 

\begin{figure}[H]
	\begin{centering}
		\includegraphics[width=0.6\textwidth]{img/smart_fix}
		\par\end{centering}
	\caption{Energy consumption based on prototypes (source: \cite{SmartFix})\label{fig:SmartFix}}
	\label{fig01c03}
\end{figure}

This paper proposed an tested new localization technology, SmartFix, which main focus is to improve energy efficiency on wearable devices. According to the experiment, probability of error within 2 meters can be reached in 80\% of cases. Meanwhile, energy consumption is 35\% lower than that of MoLoc with the same accuracy. Results show that implementing SmartFix obtains the best accuracy with minimal energy cost.

\section{Smartwatch vs. Smartphone}\label{sec:SWvsSP}
It is a comparative study about localization using smartwatch vs. smartphone \cite{SWvsSP} which presents that positioning accuracy using WiFi-based fingerprints implemented on smartwatch is sufficient for at least room-size locations. Average minimum room size is 10$m^2$ or at least 3$x$3 meters with maximum of five WiFi APs broadcasting using different channels.

Field study was conducted in six specific locations, such as farm, large room, house, two types of medium rooms and one small room. This study collected data using two Android-based devices, smartwatch MotoACTV was used and smartphone Samsung Galaxy S3 mini. To make fingerprint data more reliable the study also collects device orientation (horizontal, vertical) and data were taken in all four cardinal directions (north, east, south, west).

\begin{figure}[H]
	\begin{centering}
		\includegraphics[width=0.9\textwidth]{img/smartwatch_vs_smartphone}
		\par\end{centering}
	\caption{Summary table of results (source: \cite{SWvsSP})\label{fig:SWvsSP}}
	\label{fig02c03}
\end{figure}

\fref{fig02c03} shows the results of these experiments. Most important part of this summary is comparison of positioning accuracy between smartwatch and smartphone. In all tests the difference in classification error between the smartphone and smartwatch was at most 5\% if all five APs were used. On the other hand it can be up to 25\% worse when using small amount of APs in large environments. This study confirms that localization using smartwatch can  be comparable to smartphone and sometimes even better for home environments. The usage of smartwatches is also preferable since it is tightly connected to a person.

\chapter{Analysis, design and implementation}\label{sec:AnalysisDesignAndImplementation}
This chapter describes all important information about created application. One of the main parts are hardware and software used for developing and testing of the application. Other part is structure and description of core parts used in the application. 

\section{Hardware}\label{sec:Hardware}
There are three main Hardware components used and those are smartphone, smartwatch and BLE beacons. Main requirements for both hardwares is support for scanning Bluetooth Low Energy beacons that is supported from Bluetooth 4.0. As a secondary requirements are Wi-Fi, GSM and LTE modules to be able to scan more types of devices than just BLE beacons.  

% Maybe add beacons server information?

\subsection{Smartphone}\label{subsec:Smartphone}
Main part of the application will be developed and tested on Redmi Note 4 from Chinese company Xiaomi. It is running customized version of Android 6.0 called MIUI. Even thought system was customized in core it is still Android so there are no problems in that regard  \cite{XRN4LTE}. This phone has Bluetooth 4.1 with LE support so main requirement for the hardware is met. This device also met all the secondary requirements with Wi-Fi 802.11 a/b/g/n, GSM, and LTE support like most modern smartphones would \cite{XRN4FPS}.

\subsection{Smartwatch}\label{subsec:Smartwatch}
As already mentioned this thesis is using smartwatch with support of Android Wear 2.0 which makes it harder to select proper wear device since there not so many options at this time. There was around twenty of watches with 2.0 system and only five of them were selected to closer inspection based on few articles \cite{BAWW, BAWW18, BAWW17}.

\begin{table}[h]
	\begin{center}
		\resizebox{\textwidth}{!}{
		\begin{tabular}{| m{3.5cm} | m{2.2cm} | m{2.5cm} | l |}
			\hline
			\rowcolor[HTML]{EFEFEF}
			Watch & BLE / Wi-fi support & Sold in Czech Republic & Problems \\ \hline
			LG W280 Sport & Yes / Yes & No & \begin{tabular}[c]{@{}l@{}} Battery life is one day or less. \\ Too big in size. \end{tabular} \\ \hline
			LG W270 Titanium Style & Yes / Yes & Yes & Battery life is one day or less. \\ \hline
			Huawei Watch 2 & Yes / Yes & Yes & \begin{tabular}[c]{@{}l@{}} First update can take a long time. \\ Slight Bluetooth pairing issues. \end{tabular} \\ \hline
			Polar M600 & Yes / Yes & Yes & \begin{tabular}[c]{@{}l@{}} Polar support complains. \\ Phone synchronization issues. \\ GPS location malfunctions. \end{tabular} \\ \hline
			ASUS ZenWatch 3 & Yes / Yes & No &  \begin{tabular}[c]{@{}l@{}} Strap breaks fast. \\ AW 2.0 update can break the watch. \\ ASUS support complains. \end{tabular} \\ \hline
		\end{tabular}}
		\caption{Smartwatch comparison (sources: \cite{LGWSP, LGWST, HW2, PM600, AZW3})}
		\label{tab1}
	\end{center}
\end{table}

There were funds only for one watch device out of five which is displayed in \tref{tab1} with selection parameters. First requirement for the wear device is support of BLE and Wi-Fi which all have. Second information considered was being able to buy it in Czech Republic (CR) since it is easier for prices, shipping and reclamations. Only three of five devices were sold in CR at that time so others are out of the question. Final decision was made based on extensive research of customer reviews in shopping sites (Amazon, CZC, Heureka, Alza), wear official websites \cite{LGWSP, LGWST, HW2, PM600, AZW3} and other tech sites \cite{BAWW, BAWW18, BAWW17}. And selected device is Huawei Watch 2 since there were not too many problems in reviews and other requirements were met. 

Initial setup of the wear device was composed of two main parts. First one was update wear system that took about one to two hours. Second task was copy Google account into the wear device where problem was discovered. Copying of accounts from Redmi Note 4 ti the watch hanged nd never completed. To fix this problem another smartphone was used to copy the account but as it was already mentioned only single device can be connected to smartwatch and connecting a new one requires factory reset that would remove all the data. So there was the need to pair wear with phone without factory reset which was handled via debugging following this \cite{HtPAWW} article.

\subsection{BLE beacons}\label{subsec:BLEBeacons}
Some info about beacons.

\section{Software}\label{sec:Software}
Smart phone and wear use Android system which was already described in \hyperref[sec:Android]{Chapter 3}. This section will provide basic information about libraries, technologies and systems used in the application or supporting it.

\subsection{AltBeacon Library}\label{subsec:AltBeaconLibrary}
There are multiple solutions that can be used to scan for BLE beacons as for example Estimote SDK \cite{ESDKfA} which was already used in previous thesis \cite{PMRIL}. To change things up BLE beacons are found via AltBeacon Library \cite{ABL}.

Since there is no open and interoperable specification for proximity beacons, Radius Networks has authored the AltBeacon specification as a proposal for how to solve this problem. It is a open and free specification for Bluetooth Low Energy beacons with focus to create an open, competitive market for proximity beacon implementations \cite{AltB}.

This library enables Android devices to scan for iBeacons based on previously mentioned AltBreacon standard but it can be customized to support different kinds of beacons. It also supports Eddystone which is Google's open source beacon format and calculation of range between the devices which will help with localization \cite{ABL, EDDF}.

\subsection{Database}\label{subsec:Database}
Database is needed to keep all the Fingerprint data for calculations and there are two types of databases used for this application. First one is SQLite database that is used in Android application to save Fingerprints. This database is default and most used solution in Android applications and there is no need to use other ones. Another type of database used is Couchbase which is implemented on \verb|beacons.uhk.cz| server to keep all Fingerprint data on one place for multiple applications.

\subsubsection{Couchbase database}\label{subsec:CouchbaseDatabase}

\subsubsection{SQLite database}\label{subsec:SQLiteDatabase}
SQL (Structured Query Language) is a standard language for storing, manipulating and retrieving data in databases. It is a type of Relational Database that means all data is saved into tables with rows and columns \cite{ItSQL}. These tables usually have set amount of rows with specific names that protect from adding wrong data as en example you cannot add data \verb|Person(name, surname, eye color)| into table \verb|Person(name, surname)| because there is no column named \verb|eye color| in the table.

Advantages of these databases are structured data which makes calculation faster but uses more storage space. Data can be only saved once since they can be connected to each other. It supports complex queries for creating, reading, updating and removing data (CRUD) and better security with user and table management. Some disadvantages of this system can be with complexity and inflexibility of database scheme because it can be hard to setup and it does not allow other data then is defined in the tables \cite{ERDMS}.

Since SQL with all the features can consume a lot of hardware resources for a smartphone that is not as fast as a server Android decided to implement SQLite. SQLite has the following noticeable features: self-contained, serverless, zero-configuration, transactional \cite{WISQLITE}.

\begin{itemize}
	\item Serverless = does not need second process for the server.
	\item Self-Contained = requires minimal support from operating system.
	\item Zero-configuration = no need for installation or any configuration.
	\item Transactional = data are protected against failed changes (application crashes, power failure, ...).
\end{itemize}

This solution was tested against Couchbase database for Android and it proved as better solution. Not only it takes less storage space it is also able to load data faster. As \tref{tab2} shows SQL lite takes less data space and is almost three times faster in loading all the documents.

\begin{table}[h]
	\begin{center}
		\resizebox{\textwidth}{!}{
		\begin{tabular}{| l | l | m{3.5cm} |}
			\hline
			\rowcolor[HTML]{EFEFEF}
			Database type & Data size & Loading speed (315 documents) \\ \hline
			SQLite & 15MB & 23 second \\ \hline
			Couchbase without views & 31MB & 65 seconds \\ \hline
			Couchbase with views & 91MB & 65 seconds \\ \hline
		\end{tabular}}
		\caption{Couchbase vs SQLite (sources: \cite{LGWSP, LGWST, HW2, PM600, AZW3})}
		\label{tab2}
	\end{center}
\end{table}

\subsection{TileView}
\label{subsec:TileView}

\section{Application structure}
\label{sec:ApplicationStructure}

\subsection{Mobile application}
\label{subsec:MobileApplication}

\subsubsection{Activities}
\label{subsec:Activities}

\subsubsection{Model}
\label{subsec:Model}

\subsubsection{Utilities}
\label{subsec:Utilities}

\subsection{Wear application}
\label{subsec:WearApplication}
\chapter{Testing and data analysis}\label{sec:TestingAndDataAnalysis}

\section{Data collection}
\label{sec:DataCollection}

\section{Analysis}
\label{sec:Analysis}
\chapter{Testing and data analysis}\label{sec:TestingAndDataAnalysis}
This chapter goal is to show application testing, data collection and analysis.

\section{Testing environment}\label{sec:TestingEnvironment}
Test site of this application is a third floor of the Campus building of the Faculty of Informatics and Management, University of Hradec Kralove (FIM UHK). The main walk-through corridors are in a rectangular arrangement. Classrooms and offices are situated inwards and outwards in relation to the corridors. There is a roofed atrium in the center of the building. Experiments have been conducted in a 52 m x 43 m area.

\begin{figure}[H]
	\begin{centering}
		\includegraphics[width=0.8\textwidth]{img/j3np}
		\par\end{centering}
	\caption{Map of deployed devices (based on \cite{IILUBLEB})}
	\label{fig01c06}
\end{figure}

\fref{fig01c06} shows positions of deployed devices. There are four WiFi transmitters for \verb|eduroam| network made by Cisco (marked as W) on this floor. They are permanently placed on the ceiling and their settings could not be altered. Each marked place has usually more than one of them, typically at least two. One of them broadcasting in a 2.4 GHz and the other in a 5 GHz band. Their TX and power is automatically adjusted to help mitigate interference and signal coverage problems.

The other devices marked by numbers from 1 to 17 are Bluetooth Low Energy beacons placed evenly in the corridors and on the floor. They broadcast parameters were set to the advertising interval of 100 ms and the TX power of 0 dBm. Corridor beacons are placed in positions about 10 m apart from others \cite{IILUBLEB}. Since the beacons were placed two years ago there was a possibility of battery depletion and malfunction. By multiple tests we figured that two beacons had their battery depleted (1, 4), one was reconfigured and one was missing completely (5).

\section{Evaluation approach}\label{sec:EvaluationApproach}
Basics of evaluation are same as in previous year using solution created by Pavel Kriz in \cite{IILUBLEB}. This approach is using WKNN algorithm to compare fingerprints against each other by selecting one and comparing it to all other fingerprints. Basic premise of this evaluation is to pick one fingerprint, forget its position and try to calculate it via specific amount of neighbors. This calculated value is then compared to real one and difference between these values is the error in meters. Advantage of this approach is that it can be done in only one phase instead of two, since online phase is supplemented by one selected fingerprint. Second advantage is easy implementation and calculation of error difference since both calculated and real positions are known.

This evaluation it implemented to used on one fingerprint and compare it to others, implementing testing of BLE and WiFi signals separate or together it can provide an insight on which technology is better and if combining them improves the precision. This evaluation program also enables filtering out data not used for analysis, which are records of devices other than mentioned BLE beacons and WiFi access-points. Since this approach does not differentiate between technologies of recording devices it had to be expanded to support this.

\begin{itemize}
	\item Filtering data by the device type, which is an important part to enable test only for a specific device type. These fingerprints are tested only against its own technology and results can be used to compare precisions between technologies.
	\item Selecting fingerprint group, in this case multiple fingerprints are picked an tested. This selection is done via scan id, meaning both scan originated at the same time and place on different devices. Both fingerprints are tested individually but the results are combined together aiming to improve total localization accuracy. This approach enables two testing types, run the test against its own technology or all fingerprints.
	\item Combine data from multiple device technologies and consider it as one. Fingerprints are selected based on their scan id and all scanned data are cloned into mobile fingerprint combining them together.
\end{itemize}

All of these different approaches will enable to compare technologies and also show which approach has the lowest error so it could be used in the future. 

\subsection{WKNN}\label{sec:WKNN}
This approach is extends algorithm \textbf{K Nearest Neighbors} (KNN) with \textbf{Weights} making it WKNN. KNN is based on the idea that unclassified individual should belong to the same class as its nearest neighbor in the data set \cite{HGAfC}. It calculates the position based on the locations of selected K nearest neighbors, where K change influences the accuracy of this algorithm. Extended with weights, WKNN can make some neighbors or data more valuable than other ones. In this case, closer fingerprints have higher weight and will be used to calculate position by Euclidean distance formula. The equation for a fingerprint $f = (f_1,f_2,...,f_n)$ from the $i$th fingerprint $S_i = (s_{i1} ,s_{i2},...,s_{in})$ can be expressed by following formula \cite{IILUBLEB, HGAfC}:

$$D_i = \sqrt{\strut\sum_{j=1}^{N}(f_j-f_{ij})^2}\ ,$$

where N is a number of unique transmitters in the measurement.

These distance values are sorted and based on them first $k$ fingerprints are chosen to calculate weighted estimate of position $P$ of measured fingerprint. We can achieve this by using their known positions $P_i[x_i,y_i]$ in a following formula:

$$ P = 
\frac
	{\strut\sum_{i=1}^{k} P_i Q_i}
	{\strut\sum_{i=1}^{k} Q_i},\
where\
	Q_i = \frac{\strut 1}{\strut D_i}.$$

WKNN algorithm was chosen because it is easy to implement and result data are not difficult to interpret.

\section{Data collection}\label{sec:DataCollection}
Data was collected at 105 evenly-spaced (2m) positions of Campus building floor. Each position was measured four times, in two different headings (north, south) and with two device orientations, to prevent signal obstruction. Using both devices, 105 positions and 4 scans per position resulted in 840 different measurements consisting of 466 411 BLE and 73 757 WiFi individual RSSI samples making it in total 540 168.

Each of the scans was run for 30 seconds on both devices at the same time to provide equal environment. It is not really feasible to run such long scans in one run, so the decision was made to split sub-scans into smaller time intervals. BLE beacons have advertising interval set to 100 ms and single scan length could be set to this number but it is set to 200 ms to prevent packet loss between the scans. Since localization using WiFi usually does not need big amount of data, scanner is set to run every 5 seconds. One problem with this scanner is that Android does not allow to set scan length or cancel WiFi scanning, only thing that can be canceled is listening for new WiFi records but scan could still be running. The result of this may be that current scan receives data from previous one, this is not prevented in the application but it can be filtered out for evaluation. Cellular and sensor scanner are set to same length as WiFi since this data is used only as complimentary and it helps to reduce fingerprint data size.

Three data collection sessions were run during application development and data from the last one were used for evaluation. To consider fingerprint viable it must contain at least 20 records for Bluetooth Low Energy and WiFi, otherwise it needs to be deleted and re-scanned.

\subsection{First data collection}\label{sec:FirstDataCollection}
This first collection was meant as a test run in real environment and it consisted of scanning ten spots, each was scanned three times for 20 seconds to check if the scanner is reliable. This session revealed two main problems with the scanner.

First, with WiFi scanning on wear device where the scanner did not find any data. It was probable that there was a problem with scan length since quick 30 seconds test showed data added at the last second. This was fixed by two main changes, increasing scan length to 30 seconds and more importantly wear device now starts a WiFi scan right after the activity is displayed and before actual scanning is triggered. These changes helped to improve this scanner and it does not create any problems since there is little time between this \enquote{dummy} scan and the actual scan, usual time difference is in milliseconds.

Second, not enough data was received by BLE scanner. Data from previous years showed that each scan usually has hundreds of BLE records but this scanner was able to collect around 50 of them. It was later discovered that scan times were not properly set for AltBeacon library and scanner was run for 1 second instead of 200 milliseconds with only one device record per scan.

After both of these issues were fixed it second data collection was run to collect the data for evaluation. And there is one final thing to note, which is that wear device was not able to scan any Cellular records even though specification shows it should support LTE technology. 

\subsection{Second data collection}\label{sec:SecondDataCollection}
This collection was meant to be the final one and the data should have been used for evaluation, but there were problems revealed all the data was collected. All data seemed fine but evaluation showed problems with BLE beacons. For a start, analysis data informed that around 30 fingerprints were removed from testing due to no specific BLE beacons found. Most of these fingerprints originated on wear device.

\begin{table}[h]
	\begin{center}
		\begin{tabular}{ l C{2cm} C{2cm} C{2cm} }
			\hline
			Max error (m) & BLE & WiFi & Combined \\ 
			\hline
			All devices & 27 & 14 & 11 \\ 
			Mobile & 30 & 14 & 11 \\ 
			Wear & 31 & 11 & 11 \\ 
			\hline
		\end{tabular}
		\caption{Maximum errors for second data collection}
		\label{tab01c06}
	\end{center}
\end{table} 

\tref{tab01c06} shows a table of maximum errors in meters for collected data, as it shows BLE errors can be two or three times higher then of WiFi ones reaching almost 30 meters error which would place the device on the other side of the building. Mean and median errors showed the same trend so next step was to test if some beacons are missing in the data and display errors on the map to find the worst locations.

\begin{figure}[H]
	\begin{centering}
		\includegraphics[width=0.5\textwidth]{img/second_data_collection_errors}
		\par\end{centering}
	\caption{Map of errors for BLE in meters for all fingerprints}
	\label{fig02c06}
\end{figure}

\fref{fig02c06} shows errors on BLE scanning on the map with two indications of error size. First, circle size is calculated from error number, the higher error = bigger the circle. Second, fingerprints with error under three meter have green color and those above have color based on their error in red spectrum, more saturated red = higher error. Judging based on these information evaluation seems to be worst near beacons 1 and 4, which were later found out to be out of power. Next evaluation using differences in advertising interval revealed beacon number 14 reconfigured to different settings. Final tests were to check if all beacons are in place where number 5 was moved and had to be replaced. All of these tests resulted in four malfunctioning beacons all over the floor and deeming the data not viable to further evaluation, hence new data collection was required.

Since this was a complete data collection it also served as a test for wear device how long it can stay operational and scanning. Multiple data were collected to measure this, such as start and end time, number of places and fingerprints scanned and of course power status. Wear device is never used under 15\% because this is the threshold when power saving is turned on and device shuts down all non essential functions.

\begin{table}[h]
	\begin{center}
		\begin{tabular}{ l C{2cm} C{2cm} C{2cm} C{2cm} }
			\cline{2-3}
			& \multicolumn{2}{c}{\% Battery power} & & \\
			\hline
			Time & Start & End & Places & Fingerprints \\ 
			\hline			
			2:50 & 100\% & 15\% & 30 & 240 \\
			2:40 & 100\% & 17\% & 37 & 296 \\
			1:55 & 100\% & 35\% & 12 & 96 \\
			1:20 & 100\% & 70\% & 18 & 144 \\
			0:45 & 30\%	& 15\% & 8 & 64 \\
			\hline
		\end{tabular}
		\caption{Scanning information for wear (second scan)}
		\label{tab02c06}
	\end{center}
\end{table}

\tref{tab02c06} shows all important information for wear scanning. First thing to note when comparing the percentages of power with time is that wear device cannot usually last for more than three hours of scanning, which is no enough since all the scans took about 8 hours to complete. First two long scans show that wear can scan for around 35 places without the need for charging, which is confirmed by short scan where scanning 8 places takes around 45 minutes and 15\% of the device battery. There are also some exceptions like the third scan which was done in the middle of Campus building where there are now beacons and it was needed to re-scan multiple places many times to achieve set requirements mentioned in previous chapter.

\subsection{Third data collection}\label{sec:ThirdDataCollection}
Third and final data collection was conducted after all beacons were checked and fixed. Scanning showed only one single problem and that is again with the WiFi scanning on wear where no records are found usually after 10 spots, 80 fingerprints, were collected. This is not a huge issue that has two easy fixes. First, is restarting the device which will enable to scan for 10 more spots without problems. Second and not ideal is to turn off and on the WiFi receiver on the wear device. This fix only works for next two or three scans so it is not ideal. Both of these solutions were used based on the situation but wear was always turned on and off after 15 successful spots.

\begin{table}[h]
	\begin{center}
		\begin{tabular}{ l C{2cm} C{2cm} C{2cm} C{2cm} }
			\cline{2-3}
			& \multicolumn{2}{c}{\% Battery power} & & \\
			\hline
			Time & Start & End & Places & Fingerprints \\ 
			\hline			
			2:40 & 100\% & 15\% & 37 & 296 \\
			2:00 & 100\% & 33\% & 30 & 240 \\
			1:10 & 100\% & 58\% & 18 & 144 \\
			0:45 & 50\% & 22\% & 11 & 88 \\
			0:40 & 40\% & 15\% & 9 & 72 \\
			\hline
		\end{tabular}
		\caption{Scanning information for wear (third scan)}
		\label{tab03c06}
	\end{center}
\end{table}

For this scan only one specific change was made and that was starting \enquote{dummy} scan also for a mobile device, before this collection it was implemented only on wear device. It was changed to provide even more equal environment for both devices and make scanning procedure completely the same. Collecting of all fingerprints took about 7 hours and 15 minutes which is almost an hour less then previous which took 8 hours and 10 minutes. Battery information in \tref{tab03c06} confirm results from previous scans and show that wear device cannot scan more then three hours at the time before reaching power saving mode. This feature can be disabled after new update of wear device system but it was not used anyway since there is no guarantee it will disable all parts of power saving.

\begin{figure}[h!]
	\begin{centering}
		\includegraphics[width=0.4\textwidth]{img/third_data_collection_ble_errors}
		\includegraphics[width=0.4\textwidth]{img/third_data_collection_wifi_errors}
		\par\end{centering}
	\caption{Map of errors for BLE (left) and WiFi (right)}
	\label{fig03c06}
\end{figure}

\fref{fig02c06} shows two images of all fingerprint errors on the map thus creating eight overlapping circles on each spot. WiFi seems more precise in comparison with BLE and there might still be some problems around beacon 1. Places with one or two light red circles will usually be improved by other fingerprint on that spot but the others could increase localization error. On the other hand comparison of this figure and \fref{fig01c06} shows decrease of BLE errors for this scanning round. Comparison between WiFi and BLE suggests that WiFi should have smaller error and better accuracy but there might be issues for both technologies in the middle of the building.

\section{Evaluation}\label{sec:Evaluation}
Evaluating of all the data has multiple approaches and combinations since in needs to consider information such as algorithm implemented or multiple device types and radio signal technologies. During data collection was figured out that wear device is not able to scan for Cellular records so this radio signals will not be tested in the evaluation. First thing that needs to be decided is which $k$ variable to use with evaluations using WKNN algorithm and after that multiple tests can be run to compare device types and how data can be used to improve each other.

\subsection{Decide K values for WKNN}\label{sec:TestingKValuesForWKNN}
For selecting $k$ value all data was tested without their device information, thus putting all the data together and test them against each other. Only $k \in \{2, 3, 4\}$ were used in final testing because using $k = 1$ defeats the purpose of WKNN algorithm and setting $k > 4$ results with too big errors. Following tangle shows collection of mean, median and maximum errors in meters for all tested $k$ values.

\begin{table}[h]
	\begin{center}
		\resizebox{\textwidth}{!}{
			\begin{tabular}{ l C{2cm} C{2cm} C{2cm} | C{2cm} C{2cm} C{2cm} | C{2cm} C{2cm} C{2cm} }
				\cline{2-10}
				& \multicolumn{3}{c|}{K = 2} & \multicolumn{3}{c|}{K = 3} & \multicolumn{3}{c}{K = 4} \\
				\hline
				& BLE & WiFi & Combined & BLE & WiFi & Combined & BLE & WiFi & Combined \\ 
				\hline	
				Mean & 2.25 & 1.77 & 1.52 & 2.20 & 1.76 & 1.52 & 2.15 & 1.80 & 1.56 \\
				Median & 1.93 & 1.06 & 1.02 & 1.76 & 1.30 & 1.22 & 1.61 & 1.32 & 1.06 \\
				Maximum & 12.90 & 11.11 & 10 & 12.59 & 11.38 & 9.67 & 13.28 & 10.60 & 9.39 \\
				\hline
			\end{tabular}
		}
		\caption{List of errors for multiple K values}
		\label{tab04c06}
	\end{center}
\end{table}

Mean and median information are main decision variables and maximum is used only as a complimentary information. Considering values by technology it seems $k = 2$ is best for WiFi and $k = 4$ for BLE so it will be decided between these two. Combining technologies together shows mean and median are better for $k = 2$ and maximum for $k = 4$. Main variables are better when using $k = 2$ and that is why it was selected for next evaluations. It also achieves better accuracy with WiFi, which is more precise than BLE technology, and this value was already tested and used in previous years \cite{IILUBLEB}.

\begin{figure}[h!]
	\begin{centering}
		\includegraphics[width=0.5\textwidth]{img/wknn_errors_classic}
		\par\end{centering}
	\caption{Comparison of localization accuracy for $k = 2$}
	\label{fig04c06}
\end{figure}

\fref{fig04c06} shows comparison of both radio technologies used for evaluation and their combination. It correlates with \tref{tab04c06} showing WiFi better than BLE and their combination improved than using single of these technologies. This graph is mainly used to compare this evaluation method with future iteration of WKNN algorithm considering device technology recording fingerprints.  

\subsection{Compare device technologies}\label{sec:CompareDeviceTechnologies}
After it is knows which parameters to use evaluation of device technologies can be started. For this part minimum of data filtering was used and all the fingerprints were tested based on the device technology. To compare all data were also put together and tested against each other without taking device technology into consideration, meaning position of phone fingerprints can be calculated using wear fingerprints and vice versa. This is not the best approach since it defeats the purpose of using different device types but it is easily implemented for the first round of evaluation.

\begin{table}[h]
	\begin{center}
		\resizebox{\textwidth}{!}{
			\begin{tabular}{ l C{2cm} C{2cm} C{2cm} | C{2cm} C{2cm} C{2cm} | C{2cm} C{2cm} C{2cm} }
				\cline{2-10}
				& \multicolumn{3}{c|}{Mean error (m)} & \multicolumn{3}{c|}{Median error (m)} & \multicolumn{3}{c}{Max error (m)} \\
				\hline
				& BLE & WiFi & Combined & BLE & WiFi & Combined & BLE & WiFi & Combined \\ 
				\hline	
				Wear & 2.27 & 2.67 & 2.21 & 2.00 & 2.10 & 2.00 & 12.90 & 11.11 & 10 \\		
				Mobile & 2.05 & 0.88 & 0.85 & 1.76 & 0.80 & 0.88 & 11.50 & 8 & 6.06 \\
				All devices & 2.25 & 1.77 & 1.52 & 1.93 & 1.06 & 1.02 & 12.90 & 11.11 & 10 \\
				\hline
			\end{tabular}
		}
		\caption{Device comparison: mean and max errors (in meters)}
		\label{tab05c06}
	\end{center}
\end{table}

\tref{tab05c06} shows the comparison of device fingerprints and their calculated Mean and Max error values. BLE technology does not show much of a difference in devices, just in about 0.2 meters which makes them comparable. The WiFi, on the other hand, shows a big difference between these two types making mobile a better solution with over two times lower mean error. Fingerprint data were checked to find precise reason behind this difference.

\begin{figure}[h!]
	\begin{centering}
		\includegraphics[width=0.6\textwidth]{img/number_of_transmitters}
		\par\end{centering}
	\caption{Number of transmitters for all fingerprints}
	\label{fig05c06}
\end{figure}

\fref{fig05c06} shows number of transmitters for all fingerprints based on technology and device. It shows close numbers for BLE but there seems to be lower amount of WiFi transmitters for wear device. It seemed that wear was not able to record signals from some transmitters and after comparing data from wear to mobile it was discovered that missing transmitters use 5 GHz technology. Mobile fingerprints contain over 7 300 WiFi records originating from access-point on 5 GHz where wear has 0 of them. Thus figuring out that wear device does not contain 5 GHz WiFi receiver. Every manufacturer is in control of hardware for its devices and Huawei decided not to implement it due to high power usage. Records from 5 GHz are more precise than those of 2.4 GHz making this a reason why wear precision is not as good when comparing to mobile device.

Combining data from both wireless technologies together and comparing devices shows wear error almost three times higher than of mobile device, thus increasing the error when combining all data together. It increases mean error from mobile only 0.85 to 1.52 combined and max from 6.06 to 10, meaning using only mobile device would result in better precision instead of combining both in this case. It is connected to already mentioned WiFi precision of wear device.

\begin{figure}[h!]
	\begin{centering}
		\includegraphics[width=1\textwidth]{img/wknn_errors_mobile_phone}
		\par\end{centering}
	\caption{Comparison of errors based on device}
	\label{fig06c06}
\end{figure}

To compare both technologies more precisely all fingerprint errors were sorted are displayed in \fref{fig06c06}. The image confirms information that using wear device fingerprints will result in higher error than mobile but there is few places where using wear is better, but unfortunately there is not a lot of them. 

\subsection{Testing multiple fingerprints}\label{sec:TestingMultipleFingerprint}
Previous method is testing only single fingerprint against all other ones without the device type information. Improving upon previous design this evaluation selects multiple (two in this case) fingerprints with different device origins and runs the WKNN on them. These fingerprints have to be from the same location and also from the same scan group, which is identifies by their scan id. All of these fingerprints have their location calculated and this information is then averaged to increase the location precision. This algorithm supports multiple technologies and it could be improved to support weights where specific technology, such as mobile, could have higher weight to reduce influence of technologies with worse precision.

There are two main implementation of this evaluation. First, compares both fingerprints to all ignoring device technology like previous evaluation. Second, each of the fingerprint is compared only to its own origin device type which is used to check the influence to calculations based on data sources, thus figuring out if testing fingerprints against its own technology can improve or make the precision worse.

\begin{table}[h]
	\begin{center}
		\resizebox{\textwidth}{!}{
			\begin{tabular}{ l C{2cm} C{2cm} C{2cm} | C{2cm} C{2cm} C{2cm} }
				\cline{2-7}
				& \multicolumn{3}{c|}{Compared to own technology} & \multicolumn{3}{c}{Compared to all} \\
				\hline
				& BLE & WiFi & Combined & BLE & WiFi & Combined \\ 
				\hline	
				Mean & 2.18 & 1.77 & 1.53 & 2.30 & 1.77 & 1.52 \\
				Median & 1.87 & 1.50 & 1.33 & 1.99 & 1.50 & 1.33 \\
				Maximum & 10.41 & 6.21 & 5.78 & 10.59 & 6.21 & 5.78 \\
				\hline
			\end{tabular}
		}
		\caption{List of errors for testing multiple fingerprints}
		\label{tab06c06}
	\end{center}
\end{table}

Comparing this data with \tref{tab04c06} shows improvement in BLE precision but mean values gotten worse for WiFi and thus making combined error higher from 1.02 to 1.33 meter, which was expected due to results from previous evaluation. On the other hand there is a big decrease of maximum errors for both technologies and decreasing combined from 10 to 5.78 meters. From these values it is not really easy to compare if precision got better on not but it can be used to compare if there is any influence of comparing fingerprints with its own technology. It shows slightly better values for BLE but those do not influence combined results which are almost completely the same, meaning there is no reason to compare fingerprint only with its own technology.

\begin{figure}[h!]
	\begin{centering}
		\includegraphics[width=0.5\textwidth]{img/wknn_errors_multiple}
		\par\end{centering}
	\caption{Comparison of localization accuracy for testing multiple fingerprints}
	\label{fig07c06}
\end{figure}

When comparing this figure with \fref{fig04c06} its shows decrease of median accuracy mainly for WiFi which also reflects on combined evaluation. Overall accuracy improved because errors are more clustered together and more evenly spread, there is also decrease of maximum error from 10 to 5.78 making it more viable in complex environments.

\subsection{Combining fingerprint data}\label{sec:CombiningFingerprintData}
Last evaluations did not prove tangible improvement of localization accuracy using wear technology in combination with mobile. For this reason next evaluation takes a different approach in combining data from multiple fingerprints based on device into one. Fingerprints are grouped based scan id, which identifies them being taken at the same place and time using different devices. All the data of such fingerprints are combined into single one reducing count of data for evaluation but hopefully increasing accuracy. Since all data from fingerprints are combined the information about origin device is lost and there is no way to compare the technologies against each other.

\begin{table}[h]
	\begin{center}
		\begin{tabular}{ l C{2cm} C{2cm} C{2cm} }
			\hline
			& BLE & WiFi & Combined \\ 
			\hline	
			Mean & 2.12 & 0.93 & 0.88 \\
			Median & 1.73 & 0.87 & 0.86 \\
			Maximum & 13.99 & 7.19 & 6.99 \\
			\hline
		\end{tabular}
		\caption{List of errors for fingerprint combining}
		\label{tab07c06}
	\end{center}
\end{table}

This approach seems to be big overall improvement in accuracy with a decrease when considering maximum error for BLE. Previous evaluations proved that wear device does not provide sufficient accuracy for WiFi and this approach seems to get around that using 5 GHz records from mobile device and BLE precision is improved by the data from both devices providing better accuracy. Combining these two radio technologies results in the best overall accuracy compared to previous evaluations.

\begin{figure}[h!]
	\begin{centering}
		\includegraphics[width=0.5\textwidth]{img/wknn_errors_combined}
		\par\end{centering}
	\caption{Comparison of localization accuracy for combining fingerprints}
	\label{fig08c06}
\end{figure}

\fref{fig08c06} shows BLE error somewhere between classic approach and testing of multiple fingerprints since it has more clustered values compared to classic one but not as multiple fingerprints testing. However, WiFi shows the best localization precision and can achieve 0 meters error for multiple fingerprints. 

\subsection{Map comparison}\label{sec:MapComparison}
To confirm previous results all errors were displayed on the map to compare them and also to find problematic spots of the evaluation. These maps contain only combined error of radio technologies to make it easier to read and not complicate it with differences between BLE and WiFi. There is also one other slight change to displayed result data where errors under 200 centimeters are set to display as a circle with such error which is mainly visible in the last image of \fref{fig09c06} because there are multiple fingerprints with 0 meters error.

\begin{figure}[h!]
	\begin{centering}
		\includegraphics[width=0.4\textwidth]{img/combined_error_classic}
		\includegraphics[width=0.4\textwidth]{img/combined_error_multiple_f}
		\includegraphics[width=0.4\textwidth]{img/combined_error_f_combination}
		\par\end{centering}
	\caption{Maps of errors for all algorithms}
	\label{fig09c06}
\end{figure}

\fref{fig09c06} shows map images for all previously described and evaluated algorithms. Top left image shows errors for classic evaluation ignoring device of origin. Top right is for next round of evaluation which takes two fingerprints with different device origin, calculates their position and averages it to improve accuracy. Finally the bottom picture is for the last evaluation which combines multiple fingerprints into one. 

Classic evaluation shows a big amount of orange circles, which means that error of a specific fingerprint is over 3 meters, and most of them seem to be in the middle of Campus building where there are no beacons placed associated with this floor. The other problematic spots seems to be in the south part of the building (bottom part of the map). Rest of the places are not considered as problematic since they usually have only one fingerprint with error over 3 meters. There is 119 fingerprints with error higher than three which is around 14\%.

Moving to the second image, it shows a big improvement of the localization for all positions and filtering out which places may actually be problematic. Bringing count of fingerprints with high error from 119 to 35 which is just 4\%. As for final image, it illustrates the best precision of all algorithms with 82 fingerprints with 0 error and just 15 above 3 meters bringing it just under 2\%.   
\chapter{Conclusion}\label{sec:Conclusion}
This thesis has introduced a new way to collect radio fingerprints on mobile and wear device, smartphone and smartwatch, to improve indoor stationary localization. This solution also included changes to data distribution between devices. The system consists of server, mobile and wear devices with the Android operating system which supports Bluetooth Low Energy. This system is designed to enable creation of radio-maps and update them anytime. Evaluation of this system was based on the Weighted K-Nearest Neighbors algorithm. This evaluation work only with specifically defined beacons and WiFi access-points to maintain equal environment for all fingerprints. Based on the data acquired in a real world scenario, the results of the localization were evaluated using WiFi, BLE and their combination with addition of data from mobile, wear and their combination.

This evaluation was composed of three main algorithm implementations to figure out how to combine data from multiple device types and improve overall localization accuracy. First, testing data from each device type separate showed lower accuracy of WiFi localization on wear, this was traced back to wear device not possessing the ability to scan for 5 GHz WiFi networks. This actually makes overall localization less accurate when combining the data together, where mean error increased from 0.85 to 1.52 meters. Second, testing one fingerprint per device type and averaging them improved overall accuracy compared to previous evaluation but it is still not as precise and using single mobile device. Third and final evaluation combined data from multiple fingerprints based on device type together % TODO

Overall result of this evaluation is 

\section{Application improvements}
\label{sec:ApplicationImprovements}

\begin{thebibliography}{10}

\bibitem{GNSS}
Bernhard Hofmann-Wellenhof, Herbert Lichtenegger and Elmar Wasle. \textit{GNSS – Global Navigation Satellite Systems: GPS, GLONASS, Galileo, and more}. Springer Science \& Business Media, 2007 [cited 2018-01-10], ISBN 9783211730171.

\bibitem{LocalizationApproaches}
Xinglin Piao, Yong Zhang, Tingshu Li, Yongli Hu, Hao Liu, Ke Zhang and Yun Ge. \textit{RSS Fingerprint Based Indoor Localization Using Sparse Representation with Spatio-Temporal Constraint} [online]. National Center for Biotechnology Information, 2016 [cited 2018-01-14], Available at: \url{https://www.ncbi.nlm.nih.gov/pmc/articles/PMC5134504/}

\bibitem{PedestrianDeadReckoning}
Stéphane Beauregard and Harald Haas. \textit{Pedestrian Dead Reckoning: Basis for Personal Positioning} [online]. School of Engineering and Science
International University Bremen, 2006, Available at: \url{http://ave.dee.isep.ipp.pt/~lbf/PINSFUSION/BeHa06.pdf}

\bibitem{AaPLocalisation}
Gabriel Deak, Kevin Curran and Joan Condell. \textit{A survey of active and passive indoor localisation systems}. In: \textit{Computer Communications}. Elsevier, 2012 [cited 2018-01-11], Volume 35, Issue 16, ISSN: 0140-3664.

\bibitem{IndoorLocalizationWithoutThePain}
Krishna Chintalapudi, Anand Padmanabha Iyer, and Venkata N. Padmanabhan. \textit{Indoor Localization Without the Pain} [online]. In: Proceedings of the sixteenth annual international conference on Mobile computing and networking, 2010 [cited 2018-01-14], Available at: \url{http://dl.acm.org/citation.cfm?id=1860016}

\bibitem{RAinWILTaS}
Zahid Farid, Rosdiadee Nordin, and Mahamod Ismail. \textit{Recent Advances in Wireless Indoor Localization Techniques and System} [online]. School of Electrical, Electronics \& System Engineering, University Kebangsaan Malaysia (UKM), 2013 [cited 2018-01-15], Available at: \url{http://downloads.hindawi.com/journals/jcnc/2013/185138.pdf}

\bibitem{LTinWSN}
Shweta Singh, Ravi Shakya and Yaduvir Singh. \textit{Localization techniques in wireless sensor networks} [online]. Department of Computer Science,
Ideal Institute of Technology, Ghaziabad, 2015 [cited 2018-01-15], ISSN: 0975-9646, Available at: \url{https://pdfs.semanticscholar.org/6299/85defbf9cc1a937a1b88c9c2a893552e3d89.pdf}

\bibitem{AoALforWSN}
Paweł Kułakowski, Javier Vales-Alonso, Esteban Egea-López, Wiesław Ludwin and Joan García-Haro. \textit{Angle-of-arrival localization based on antenna arrays for wireless sensor networks} [online]. In: \textit{Computers \& Electrical Engineering}. Elsevier, 2010 [cited 2018-01-15], Volume 36, Issue 6, Pages 1181-1186. Available at: \url{http://ai2-s2-pdfs.s3.amazonaws.com/17c6/0e17c4e72cc3fd821e12169c1c2ca7736bd4.pdf}

\bibitem{TvTHGPSRW}
GISGeography. \textit{Trilateration vs Triangulation – How GPS Receivers Work} [online]. GISGeography.com, 2018 [cited 2018-01-15]. Available at: \url{http://gisgeography.com/trilateration-triangulation-gps/}

\end{thebibliography}

%\addcontentsline{toc}{chapter}{Bibliography}
\cleardoublepage{}

% Include attachments
\addchap{Attachments}\label{sec:Attachments}
\begin{enumerate}
	\item CD containing following content
	\begin{enumerate}
		\item /applications - contains data for all applications created
		\begin{enumerate}
			\item /mobile\_wear - source code for android applications
			\item /server - source code for server application
		\end{enumerate}
		\item /evaluation - results of the analysis and data used for it
		\begin{enumerate}
			\item /gnuplot\_scripts - gnuplot scripts used to draw data using graphs
			\item /images - images created during evaluation
			\item /raw\_data - all data from the evaluation in raw format
		\end{enumerate}
		\item /tex - thesis text in tex format
	\end{enumerate}
	\item Public Github repository with Android applications (mobile and wear) and thesis text:
	\\ \url{https://github.com/Del-S/WearFingerprint}
	\item Public Github repository with server application:
	\\ \url{https://github.com/Del-S/WearFingerprintServer}
	\item Public Github repository with evaluation code in a separate branch for this project:
	\\ \url{https://github.com/pavkriz/radio-localization-eval/tree/sucharda}
\end{enumerate}

% Include pdf scan of Thesis entry
\includepdf[pages={1}]{assignment.pdf}

\end{document}