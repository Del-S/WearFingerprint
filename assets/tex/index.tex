%% LyX 1.5.5 created this file.  For more info, see http://www.lyx.org/.
%% Do not edit unless you really know what you are doing.
\documentclass[a4paper,english,english,openright,cleardoubleempty,BCOR10mm,DIV11,12pt]{scrreprt}
\usepackage[T1]{fontenc}
\usepackage[utf8]{inputenc}
\usepackage[english]{babel}
\usepackage{array}
\usepackage{float}
\usepackage{longtable}
\usepackage{varioref}
\usepackage{wrapfig}
\usepackage{fancybox}
\usepackage{calc}
\usepackage{framed}
\usepackage{url}
\def\UrlBreaks{\do\/\do-}
\usepackage{graphicx}
\usepackage{placeins} %floatbarrier \FloatBarrier
%\usepackage{listing}
\usepackage{pdfpages}
\usepackage{epstopdf}
\usepackage[left=3cm,top=2.5cm,right=2.5cm,bottom=2.5cm]{geometry}
\usepackage{breakurl}
\usepackage{indentfirst}

% Cut words in half
%\hyphenation{Word-cut}

\makeatletter

\usepackage[font=small,labelfont=bf]{caption} %captiony

%%%%%%%%%%%%%%%%%%%%%%%%%%%%%% LyX specific LaTeX commands.
\providecommand{\LyX}{L\kern-.1667em\lower.25em\hbox{Y}\kern-.125emX\@}
\newcommand{\lyxline}[1][1pt]{%
  \par\noindent%
  \rule[.5ex]{\linewidth}{#1}\par}
\newcommand{\noun}[1]{\textsc{#1}}
%% Special footnote code from the package 'stblftnt.sty'
%% Author: Robin Fairbairns -- Last revised Dec 13 1996
\let\SF@@footnote\footnote
\def\footnote{\ifx\protect\@typeset@protect
    \expandafter\SF@@footnote
  \else
    \expandafter\SF@gobble@opt
  \fi
}

\renewcommand{\baselinestretch}{1.5} %line height

\expandafter\def\csname SF@gobble@opt \endcsname{\@ifnextchar[%]
  \SF@gobble@twobracket
  \@gobble
}
\edef\SF@gobble@opt{\noexpand\protect
  \expandafter\noexpand\csname SF@gobble@opt \endcsname}
\def\SF@gobble@twobracket[#1]#2{}
%% Because html converters don't know tabularnewline
\providecommand{\tabularnewline}{\\}

%%%%%%%%%%%%%%%%%%%%%%%%%%%%%% Textclass specific LaTeX commands.
\newenvironment{lyxcode}
{\begin{list}{}{
\setlength{\rightmargin}{\leftmargin}
\setlength{\listparindent}{0pt}% needed for AMS classes
\raggedright
\setlength{\itemsep}{0pt}
\setlength{\parsep}{0pt}
\normalfont\ttfamily}%
 \item[]}
{\end{list}}

%%%%%%%%%%%%%%%%%%%%%%%%%%%%%% User specified LaTeX commands.
%<-------------------------------společná nastavení------------------------------>
%s\usepackage[]{babel}%počeštění názvů (Obsah, Kapitola, Literatura atp.)
\usepackage[]{hyperref} %odkazy v  pdf jsou klikací s barevnými rámečky
\usepackage[numbers,sort&compress]{natbib} %balíček pro citace literatury
%\usepackage{hypernat}%interakce mezi hyperref a natbib
%\newcommand{\BibTeX}{{\sc Bib}\TeX}%BibTeX logo
\hypersetup{   % Nastavení polí PDF dokumentu
pdftitle={Radio Fingerprint Acquisition Using a SmartWatch},%
pdfauthor={Bc. David Sucharda},%
pdfsubject={},%
pdfkeywords={Fingerprint, Android, Wear}%
}
\usepackage{multicol}




%<-----------------------------volání stylů----------------------------------------->
% (znak % je označení komentáře: co je za ním, není aktivní)
%<------------------------------------písmo----------------------------------------->
%\usepackage{pkg/bc-latinmodern}
%\usepackage{pkg/bc-times}
\usepackage{pkg/bc-palatino}
%\usepackage{pkg/bc-iwona}
%\usepackage{pkg/bc-helvetika}


%<------------------------------záhlaví stránek------------------------------------>
%\usepackage{pkg/bc-headings}
\usepackage{pkg/bc-fancyhdr}

%<------------------------------hlavičky kapitol------------------------------------>
%\usepackage{pkg/bc-neueskapitel}
%\usepackage{pkg/bc-fancychap}

\makeatother

\usepackage{babel}

%java code block%

\usepackage{listing}
\usepackage{listings}
\usepackage{color}

\definecolor{dkgreen}{rgb}{0,0.6,0}
\definecolor{gray}{rgb}{0.5,0.5,0.5}
\definecolor{mauve}{rgb}{0.58,0,0.82}

\renewcommand*{\lstlistingname}{Code example} %prejmenovani
\renewcommand*{\lstlistlistingname}{List of code examples}

% syntax highlight pro jazyk Java %
\lstset{
  %frame=r,
  captionpos=b,
  language=Java,
  aboveskip=3mm,
  belowskip=3mm,
  xleftmargin=0.2mm,
  showstringspaces=false,
  columns=flexible,
  basicstyle={\small\ttfamily},
  numbers=none,
  numberstyle=\tiny\color{gray},
  keywordstyle=\color{blue},
  commentstyle=\color{dkgreen},
  stringstyle=\color{mauve},
  breaklines=true,
  breakatwhitespace=true,
  tabsize=3,
    inputencoding=utf8,
    extendedchars=true,
    literate=%
    {á}{{\'a}}1
    {č}{{\v{c}}}1
    {ď}{{\v{d}}}1
    {é}{{\'e}}1
    {ě}{{\v{e}}}1
    {í}{{\'i}}1
    {ň}{{\v{n}}}1
    {ó}{{\'o}}1
    {ř}{{\v{r}}}1
    {š}{{\v{s}}}1
    {ť}{{\v{t}}}1
    {ú}{{\'u}}1
    {ů}{{\r{u}}}1
    {ý}{{\'y}}1
    {ž}{{\v{z}}}1
    {Á}{{\'A}}1
    {Č}{{\v{C}}}1
    {Ď}{{\v{D}}}1
    {É}{{\'E}}1
    {Ě}{{\v{E}}}1
    {Í}{{\'I}}1
    {Ň}{{\v{N}}}1
    {Ó}{{\'O}}1
    {Ř}{{\v{R}}}1
    {Š}{{\v{S}}}1
    {Ť}{{\v{T}}}1
    {Ú}{{\'U}}1
    {Ů}{{\r{U}}}1
    {Ý}{{\'Y}}1
    {Ž}{{\v{Z}}}1
}

\begin{document}
\renewcommand{\figurename}{Picture}
\renewcommand{\tablename}{Table}
\renewcommand{\contentsname}{Content}
\renewcommand{\bibname}{Literature}
\renewcommand{\listfigurename}{List of pictures}
\renewcommand{\listtablename}{List of tables}
%~\thispagestyle{empty}{\small ~\vfill{}
%}{\small \par}

%~\thispagestyle{empty}\vfill{}
%Tato stránka je tzv. protititul a je graficky součástí titulní stránky.
%Nechte ji prázdnou, nebo na ni umístěte vhodnou fotografii či ilustraci.

\cleardoublepage{}~\thispagestyle{empty}\begin{center}\pagenumbering{roman}\vspace{10mm}


\textsf{\textsc{\noun{\LARGE University of Hradec Králové}}}\\
\vspace{0.5em}
\textsc{\noun{\LARGE Faculty of Informatics and Management}}\\
\vspace*{1em}
\textsf{\textsc{\noun{\Large Department of Information Technologies }}}

\vspace{15mm}

%\includegraphics[width=0.4\textwidth]{img/logo_uhk}

\vspace{15mm}


\textsf{\huge MASTER'S THESIS}{\huge \par}

\vspace{15mm}


\textsf{\LARGE Radio Fingerprint Acquisition Using a SmartWatch}{\LARGE \par}

\vspace{10mm}


\end{center}

\vspace*{\fill}


\vspace{10mm}


\begin{description}
\item [{{\large Author:}}] \noindent \textsf{\large Bc. David Sucharda}{\large \par}
\item [{{\large Study programme:}}] \noindent \textsf{\large Applied Informatics}{\large \par}
\item [{{\large Supervisor:}}] \noindent \textsf{\large Ing. Pavel Kříž, Ph.D.}

{\large \bigskip{}
}\noindent {\large{} Hradec Králové \hspace{\fill}April 2018}\\
{\large{}
}{\large \par}

\end{description}
\clearpage{}

%{\small \thispagestyle{plain}\addcontentsline{toc}{chapter}{Abstrakt} }{\small \par}

\newpage{}\thispagestyle{empty}

{\small %\setcounter{page}{3} % nastavení číslování stránek
\ }{\small \par}

\noindent {\small \vfill{}
 % nastavuje dynamické umístění následujícího textu do spodní části stránky
~}{\small \par}


% Declaration in Czech
\paragraph{Prohlášení}

\noindent {\small \\Prohlašuji, že jsem diplomovou práci vypracoval samostatně a uvedl jsem všechny použité prameny a literaturu.}{\small \par}
\vspace{5mm}

% Declaration in English
\paragraph{Declaration}

\noindent {\small \\I declare that I have elaborated this thesis independently and listed all the sources and literature.}{\small \par}
\vspace{25mm}

% Sign text
{\small \bigskip{}
}\noindent {\small{} Hradec Králové day 26th of April 2018\hspace{\fill}Bc. David Sucharda}\\
{\small{} % doplňte patřičné datum, jméno a příjmení
}{\small \par}

\clearpage{}

\newpage{}\thispagestyle{empty}

{\small %\setcounter{page}{3} % nastavení číslování stránek
\ }{\small \par}

\noindent {\small \vfill{}
 % nastavuje dynamické umístění následujícího textu do spodní části stránky
~}{\small \par}

\paragraph{Poděkování}

\noindent {\small \\Rád bych zde poděkoval Ing. Pavlu Kříži, Ph.D. za odborné vedení práce, podnětné rady a čas, který mi věnoval.}{\small \par}
\vspace{5mm}

\paragraph{Thanks}

\noindent {\small \\I would like to thank to Ing. Pavel Křiž, Ph.D. for professional guidance, incentive advices, and the time he gave me.\newpage{}}{\small \par}

\clearpage{}

\newpage{}\thispagestyle{empty}

{\small %\setcounter{page}{3} % nastavení číslování stránek
\ }{\small \par}

\noindent {\small \vfill{}
 % nastavuje dynamické umístění následujícího textu do spodní části stránky
~}{\small \par}

\paragraph{Anotace}
\noindent\textbf{\small \\Název práce: Sběr ráriových fingerprintů pomocí chytrých hodinek}{\small \par}
\noindent \\Text.

\paragraph{Annotation}
\noindent \\Text

\cleardoublepage{}

%\thispagestyle{empty}~{\small \addcontentsline{toc}{chapter}{Zadání
%práce} }{\small \par}

{\small %%%   Výtisk pak na tomto míste nezapomeňte PODEPSAT!
%%%                                         *********
}{\small \par}

\cleardoublepage{}\thispagestyle{empty}{\small
%\setcounter{secnumdepth}{3}
\setcounter{tocdepth}{3}%hloubla obsahu
\pagenumbering{gobble}
\tableofcontents{}% vkládá automaticky generovaný obsah dokumentu
\listoffigures{}
\listoftables{}

\thispagestyle{empty}

\chapter{Introduction}\label{sec:Introduction}
\pagenumbering{arabic}
\setcounter{page}{1}
As the technology evolves it unlocks more and more possibilities. Just a few years back there were no smartwatches or phones but at this time they are important part of our lives. As they evolve there is the need for them to have more functions and features. One of these features is to be able to locate its position on the map. This information is very useful since it can prevent people from getting lost, figuring out path to drive, used by military and in countless more cases.

Finding such position is possible using Global Navigation Satellite System (GNSS). Multiple implementations of this system exist, such as GPS, GLONASS or Galileo. All of these systems provide location using sufficient number (at least four) of satellites \cite{GNSS, GNSSGPS}. GNSS solution requires clear line of sight between satellites and the receiving device because signal is not able to pass through buildings. This makes it the main reason why it cannot be used for indoor localization.

\begin{figure}[h!]
	\begin{centering}
		\includegraphics[width=0.7\textwidth]{img/1_comparison_of_positionin_technologies}
		\par\end{centering}
	\caption{Comparison of Positioning Technologies (source: \cite{PedestrianDeadReckoning})\label{fig:1_comparison_of_positionin_technologies}}
	\label{fig01c01}
\end{figure}

There are multiple approaches to find out location inside the building. They can be divided into three main types. First, using wireless signal ranging approach with multiple kinds of data such as Time of Arrival (ToA). Second, using special equipment like active bats (Ultrasonic). Final, based on Signal Strength Fingerprint Maps (SSFM), in which first part is to collect signal strengths from the environment and construct fingerprint maps. These maps are then used to match with current signal to obtain device location \cite{LocalizationApproaches}.

In addition to these types of localization there are also multiple algorithms used in indoor environments. Some of them are location fingerprinting, triangulation, proximity and dead reckoning \cite{AaPLocalisation}. Few of these algorithms will be described in \hyperref[sec:LocalizationTechniques]{Chapter 2}.

This thesis is focused on method using radio signal strength (RSS) fingerprinting by collecting data from Bluetooth, wireless and cellular devices.

\section{Goals of this thesis}\label{sec:GoalsOfThisThesis}
Main goal of this thesis is to explore possibilities of fingerprint acquisition using smartwatch technology. The first question that needs to be answered is if this can be done. Is smartwatch capable of RSS data collection? And the answer to this question is yes, since smartwatches have the similar specifications as low-end smartphones, containing Bluetooth and WiFi chips, which means it can be done. 

One of the goals for this thesis is to create an application for Android phone and wear device which handles RSS fingerprint collection. Problem with smartwatches is their diversity in operational systems because a lot of watch creators have their own custom systems which can complicate things. Luckily there is a version of Android operating system called WearOS (used to be Wear 2.0) and it is basically a port of Android system to wearable devices. 

Final goal is to test created application and figure out if data from wear device increase precision of indoor localization or not.

\section{Reason for selection of this topic}\label{sec:ReasonForSelectionOfThisTopic}
The reason behind selection of this topic is rather simple. I was introduced to Android during my studies at the University Hradec Králové but it was just basic knowledge. That is why I later decided to go for a study abroad to deepen my knowledge. Part of that study was to work for a selected company where we developed rather complex Android application. Its core part was using multiple APIs, user authentication and data encryption but it was still focused only on a singe device, that being the phone. So next thing I wanted to try was working with multiple kinds of devices and since WearOS is rather new I wanted to test it out. So the main reason is to get more experienced with Android and as a developer.


\begin{thebibliography}{10}

\bibitem{GNSS}
Bernhard Hofmann-Wellenhof, Herbert Lichtenegger and Elmar Wasle. \textit{GNSS – Global Navigation Satellite Systems: GPS, GLONASS, Galileo, and more}. Springer Science \& Business Media, 2007 [cited 2018-01-10], ISBN 9783211730171.

\bibitem{LocalizationApproaches}
Xinglin Piao, Yong Zhang, Tingshu Li, Yongli Hu, Hao Liu, Ke Zhang and Yun Ge. \textit{RSS Fingerprint Based Indoor Localization Using Sparse Representation with Spatio-Temporal Constraint} [online]. National Center for Biotechnology Information, 2016 [cited 2018-01-14], Available at: \url{https://www.ncbi.nlm.nih.gov/pmc/articles/PMC5134504/}

\bibitem{PedestrianDeadReckoning}
Stéphane Beauregard and Harald Haas. \textit{Pedestrian Dead Reckoning: Basis for Personal Positioning} [online]. School of Engineering and Science
International University Bremen, 2006, Available at: \url{http://ave.dee.isep.ipp.pt/~lbf/PINSFUSION/BeHa06.pdf}

\bibitem{AaPLocalisation}
Gabriel Deak, Kevin Curran and Joan Condell. \textit{A survey of active and passive indoor localisation systems}. In: \textit{Computer Communications}. Elsevier, 2012 [cited 2018-01-11], Volume 35, Issue 16, ISSN: 0140-3664.

\bibitem{IndoorLocalizationWithoutThePain}
Krishna Chintalapudi, Anand Padmanabha Iyer, and Venkata N. Padmanabhan. \textit{Indoor Localization Without the Pain} [online]. In: Proceedings of the sixteenth annual international conference on Mobile computing and networking, 2010 [cited 2018-01-14], Available at: \url{http://dl.acm.org/citation.cfm?id=1860016}

\bibitem{RAinWILTaS}
Zahid Farid, Rosdiadee Nordin, and Mahamod Ismail. \textit{Recent Advances in Wireless Indoor Localization Techniques and System} [online]. School of Electrical, Electronics \& System Engineering, University Kebangsaan Malaysia (UKM), 2013 [cited 2018-01-15], Available at: \url{http://downloads.hindawi.com/journals/jcnc/2013/185138.pdf}

\bibitem{LTinWSN}
Shweta Singh, Ravi Shakya and Yaduvir Singh. \textit{Localization techniques in wireless sensor networks} [online]. Department of Computer Science,
Ideal Institute of Technology, Ghaziabad, 2015 [cited 2018-01-15], ISSN: 0975-9646, Available at: \url{https://pdfs.semanticscholar.org/6299/85defbf9cc1a937a1b88c9c2a893552e3d89.pdf}

\bibitem{AoALforWSN}
Paweł Kułakowski, Javier Vales-Alonso, Esteban Egea-López, Wiesław Ludwin and Joan García-Haro. \textit{Angle-of-arrival localization based on antenna arrays for wireless sensor networks} [online]. In: \textit{Computers \& Electrical Engineering}. Elsevier, 2010 [cited 2018-01-15], Volume 36, Issue 6, Pages 1181-1186. Available at: \url{http://ai2-s2-pdfs.s3.amazonaws.com/17c6/0e17c4e72cc3fd821e12169c1c2ca7736bd4.pdf}

\bibitem{TvTHGPSRW}
GISGeography. \textit{Trilateration vs Triangulation – How GPS Receivers Work} [online]. GISGeography.com, 2018 [cited 2018-01-15]. Available at: \url{http://gisgeography.com/trilateration-triangulation-gps/}

\end{thebibliography}


\addcontentsline{toc}{chapter}{Literature}
\cleardoublepage{}

%\include{prilohy}

% Include pdf scan of Thesis entry
%\includepdf[pages={1}]{zadani.pdf}

\end{document}