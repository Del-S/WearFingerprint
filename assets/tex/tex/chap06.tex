\chapter{Testing and data analysis}\label{sec:TestingAndDataAnalysis}
This chapter goal is to show application testing, data collection and analysis.

\section{Testing environment}\label{sec:TestingEnvironment}
Test site of this application is a third floor of the Campus building of the Faculty of Informatics and Management, University of Hradec Kralove (FIM UHK). The main walk-through corridors are in a rectangular arrangement. Classrooms and offices are situated inwards and outwards in relation to the corridors. There is a roofed atrium in the center of the building. Experiments have been conducted in a 52 m x 43 m area.

\begin{figure}[H]
	\begin{centering}
		\includegraphics[width=0.8\textwidth]{img/j3np}
		\par\end{centering}
	\caption{Map of deployed devices (based on \cite{IILUBLEB})}
	\label{fig01c06}
\end{figure}

\fref{fig01c06} shows positions of deployed devices. There are four WiFi transmitters for \verb|eduroam| network made by Cisco (marked as W) on this floor. They are permanently placed on the ceiling and their settings could not be altered. Each marked place has usually more than one of them, typically at least two. One of them broadcasting in a 2.4 GHz and the other in a 5 GHz band. Their TX and power is automatically adjusted to help mitigate interference and signal coverage problems.

The other devices marked by numbers from 1 to 17 are Bluetooth Low Energy beacons placed evenly in the corridors and on the floor. They broadcast parameters were set to the advertising interval of 100 ms and the TX power of 0 dBm. Corridor beacons are placed in positions about 10 m apart from others \cite{IILUBLEB}. Since the beacons were placed two years ago there was a possibility of battery depletion and malfunction. By multiple tests we figured that two beacons had their battery depleted (1, 4), one was reconfigured and one was missing completely (5).

\section{Evaluation approach}\label{sec:EvaluationApproach}
Basics of evaluation are same as in previous year using solution created by Pavel Kriz in \cite{IILUBLEB}. This approach is using WKNN algorithm to compare fingerprints against each other by selecting one and comparing it to all other fingerprints. Basic premise of this evaluation is to pick one fingerprint, forget its position and try to calculate it via specific amount of neighbors. This calculated value is then compared to real one and difference between these values is the error in meters. Advantage of this approach is that it can be done in only one phase instead of two, since online phase is supplemented by one selected fingerprint. Second advantage is easy implementation and calculation of error difference since both calculated and real positions are known.

This evaluation it implemented to used on one fingerprint and compare it to others, implementing testing of BLE and WiFi signals separate or together it can provide an insight on which technology is better and if combining them improves the precision. This evaluation program also enables filtering out data not used for analysis, which are records of devices other than mentioned BLE beacons and WiFi access-points. Since this approach does not differentiate between technologies of recording devices it had to be expanded to support this.

\begin{itemize}
	\item Filtering data by the device type, which is an important part to enable test only for a specific device type. These fingerprints are tested only against its own technology and results can be used to compare precisions between technologies.
	\item Selecting fingerprint group, in this case multiple fingerprints are picked an tested. This selection is done via scan id, meaning both scan originated at the same time and place on different devices. Both fingerprints are tested individually but the results are combined together aiming to improve total localization accuracy. This approach enables two testing types, run the test against its own technology or all fingerprints.
	\item Combine data from multiple device technologies and consider it as one. Fingerprints are selected based on their scan id and all scanned data are cloned into mobile fingerprint combining them together.
\end{itemize}

All of these different approaches will enable to compare technologies and also show which approach has the lowest error so it could be used in the future. 

\subsection{WKNN}\label{sec:WKNN}
This approach is extends algorithm \textbf{K Nearest Neighbors} (KNN) with \textbf{Weights} making it WKNN. KNN is based on the idea that unclassified individual should belong to the same class as its nearest neighbor in the data set \cite{HGAfC}. It calculates the position based on the locations of selected K nearest neighbors, where K change influences the accuracy of this algorithm. Extended with weights, WKNN can make some neighbors or data more valuable than other ones. In this case, closer fingerprints have higher weight and will be used to calculate position by Euclidean distance formula. The equation for a fingerprint $f = (f_1,f_2,...,f_n)$ from the $i$th fingerprint $S_i = (s_{i1} ,s_{i2},...,s_{in})$ can be expressed by following formula \cite{IILUBLEB, HGAfC}:

$$D_i = \sqrt{\strut\sum_{j=1}^{N}(f_j-f_{ij})^2}\ ,$$

where N is a number of unique transmitters in the measurement.

These distance values are sorted and based on them first $k$ fingerprints are chosen to calculate weighted estimate of position $P$ of measured fingerprint. We can achieve this by using their known positions $P_i[x_i,y_i]$ in a following formula:

$$ P = 
\frac
	{\strut\sum_{i=1}^{k} P_i Q_i}
	{\strut\sum_{i=1}^{k} Q_i},\
where\
	Q_i = \frac{\strut 1}{\strut D_i}.$$

WKNN algorithm was chosen because it is easy to implement and result data are not difficult to interpret.

\section{Data collection}\label{sec:DataCollection}
Data was collected at 105 evenly-spaced (2m) positions of Campus building floor. Each position was measured four times, in two different headings (north, south) and with two device orientations, to prevent signal obstruction. Using both devices, 105 positions and 4 scans per position resulted in 840 different measurements consisting of 466 411 BLE and 73 757 WiFi individual RSSI samples making it in total 540 168.

Each of the scans was run for 30 seconds on both devices at the same time to provide equal environment. It is not really feasible to run such long scans in one run, so the decision was made to split sub-scans into smaller time intervals. BLE beacons have advertising interval set to 100 ms and single scan length could be set to this number but it is set to 200 ms to prevent packet loss between the scans. Since localization using WiFi usually does not need big amount of data, scanner is set to run every 5 seconds. One problem with this scanner is that Android does not allow to set scan length or cancel WiFi scanning, only thing that can be canceled is listening for new WiFi records but scan could still be running. The result of this may be that current scan receives data from previous one, this is not prevented in the application but it can be filtered out for evaluation. Cellular and sensor scanner are set to same length as WiFi since this data is used only as complimentary and it helps to reduce fingerprint data size.

Three data collection sessions were run during application development and data from the last one were used for evaluation. To consider fingerprint viable it must contain at least 20 records for Bluetooth Low Energy and WiFi, otherwise it needs to be deleted and re-scanned.

\subsection{First data collection}\label{sec:FirstDataCollection}
This first collection was meant as a test run in real environment and it consisted of scanning ten spots, each was scanned three times for 20 seconds to check if the scanner is reliable. This session revealed two main problems with the scanner.

First, with WiFi scanning on wear device where the scanner did not find any data. It was probable that there was a problem with scan length since quick 30 seconds test showed data added at the last second. This was fixed by two main changes, increasing scan length to 30 seconds and more importantly wear device now starts a WiFi scan right after the activity is displayed and before actual scanning is triggered. These changes helped to improve this scanner and it does not create any problems since there is little time between this \enquote{dummy} scan and the actual scan, usual time difference is in milliseconds.

Second, not enough data was received by BLE scanner. Data from previous years showed that each scan usually has hundreds of BLE records but this scanner was able to collect around 50 of them. It was later discovered that scan times were not properly set for AltBeacon library and scanner was run for 1 second instead of 200 milliseconds with only one device record per scan.

After both of these issues were fixed it second data collection was run to collect the data for evaluation.

\subsection{Second data collection}\label{sec:SecondDataCollection}
This collection was meant to be the final one and the data should have been used for evaluation, but there were problems revealed all the data was collected. All data seemed fine but evaluation showed problems with BLE beacons. For a start, analysis data informed that around 30 fingerprints were removed from testing due to no specific BLE beacons found. Most of these fingerprints originated on wear device.

\begin{table}[h]
	\begin{center}
		\begin{tabular}{ l C{2cm} C{2cm} C{2cm} }
			\hline
			Max error (m) & BLE & WiFi & Combined \\ 
			\hline
			All devices & 27 & 14 & 11 \\ 
			Mobile & 30 & 14 & 11 \\ 
			Wear & 31 & 11 & 11 \\ 
			\hline
		\end{tabular}
		\caption{Maximum errors for second data collection}
		\label{tab01c06}
	\end{center}
\end{table} 

\tref{tab01c06} shows a table of maximum errors in meters for collected data, as it shows BLE errors can be two or three times higher then of WiFi ones reaching almost 30 meters error which would place the device on the other side of the building. Mean and median errors showed the same trend so next step was to test if some beacons are missing in the data and display errors on the map to find the worst locations.

\begin{figure}[H]
	\begin{centering}
		\includegraphics[width=0.5\textwidth]{img/second_data_collection_errors}
		\par\end{centering}
	\caption{Map of errors for BLE in meters for all fingerprints}
	\label{fig02c06}
\end{figure}

\fref{fig02c06} shows errors on BLE scanning on the map with two indications of error size. First, circle size is calculated from error number, the higher error = bigger the circle. Second, fingerprints with error under three meter have green color and those above have color based on their error in red spectrum, more saturated red = higher error. Judging based on these information evaluation seems to be worst near beacons 1 and 4, which were later found out to be out of power. Next evaluation using differences in advertising interval revealed beacon number 14 reconfigured to different settings. Final tests were to check if all beacons are in place where number 5 was moved and had to be replaced. All of these tests resulted in four malfunctioning beacons all over the floor and deeming the data not viable to further evaluation, hence new data collection was required.

Since this was a complete data collection it also served as a test for wear device how long it can stay operational and scanning. Multiple data were collected to measure this, such as start and end time, number of places and fingerprints scanned and of course power status. Wear device is never used under 15\% because this is the threshold when power saving is turned on and device shuts down all non essential functions.

\begin{table}[h]
	\begin{center}
		\begin{tabular}{ l C{2cm} C{2cm} C{2cm} C{2cm} }
			\cline{2-3}
			& \multicolumn{2}{c}{\% Battery power} & & \\
			\hline
			Time & Start & End & Places & Fingerprints \\ 
			\hline			
			2:50 & 100\% & 15\% & 30 & 240 \\
			2:40 & 100\% & 17\% & 37 & 296 \\
			1:55 & 100\% & 35\% & 12 & 96 \\
			1:20 & 100\% & 70\% & 18 & 144 \\
			0:45 & 30\%	& 15\% & 8 & 64 \\
			\hline
		\end{tabular}
		\caption{Scanning information for wear (second scan)}
		\label{tab02c06}
	\end{center}
\end{table}

\tref{tab02c06} shows all important information for wear scanning. First thing to note when comparing the percentages of power with time is that wear device cannot usually last for more than three hours of scanning, which is no enough since all the scans took about 8 hours to complete. First two long scans show that wear can scan for around 35 places without the need for charging, which is confirmed by short scan where scanning 8 places takes around 45 minutes and 15\% of the device battery. There are also some exceptions like the third scan which was done in the middle of Campus building where there are now beacons and it was needed to re-scan multiple places many times to achieve set requirements mentioned in previous chapter.

\subsection{Third data collection}\label{sec:ThirdDataCollection}
Third and final data collection was conducted after all beacons were checked and fixed. Scanning showed only one single problem and that is again with the WiFi scanning on wear where no records are found usually after 10 spots, 80 fingerprints, were collected. This is not a huge issue that has two easy fixes. First, is restarting the device which will enable to scan for 10 more spots without problems. Second and not ideal is to turn off and on the WiFi receiver on the wear device. This fix only works for next two or three scans so it is not ideal. Both of these solutions were used based on the situation but wear was always turned on and off after 15 successful spots.

\begin{table}[h]
	\begin{center}
		\begin{tabular}{ l C{2cm} C{2cm} C{2cm} C{2cm} }
			\cline{2-3}
			& \multicolumn{2}{c}{\% Battery power} & & \\
			\hline
			Time & Start & End & Places & Fingerprints \\ 
			\hline			
			2:40 & 100\% & 15\% & 37 & 296 \\
			2:00 & 100\% & 33\% & 30 & 240 \\
			1:10 & 100\% & 58\% & 18 & 144 \\
			0:45 & 50\% & 22\% & 11 & 88 \\
			0:40 & 40\% & 15\% & 9 & 72 \\
			\hline
		\end{tabular}
		\caption{Scanning information for wear (third scan)}
		\label{tab03c06}
	\end{center}
\end{table}

For this scan only one specific change was made and that was starting \enquote{dummy} scan also for a mobile device, before this collection it was implemented only on wear device. It was changed to provide even more equal environment for both devices and make scanning procedure completely the same. Collecting of all fingerprints took about 7 hours and 15 minutes which is almost an hour less then previous which took 8 hours and 10 minutes. Battery information in \tref{tab03c06} confirm results from previous scans and show that wear device cannot scan more then three hours at the time before reaching power saving mode. This feature can be disabled after new update of wear device system but it was not used anyway since there is no guarantee it will disable all parts of power saving.

\begin{figure}[h!]
	\begin{centering}
		\includegraphics[width=0.4\textwidth]{img/third_data_collection_ble_errors}
		\includegraphics[width=0.4\textwidth]{img/third_data_collection_wifi_errors}
		\par\end{centering}
	\caption{Map of errors for BLE (left) and WiFi (right)}
	\label{fig03c06}
\end{figure}

\fref{fig02c06} shows two images of all fingerprint errors on the map thus creating eight overlapping circles on each spot. WiFi seems more precise in comparison with BLE and there might still be some problems around beacon 1. Places with one or two light red circles will usually be improved by other fingerprint on that spot but the others could increase localization error. On the other hand comparison of this figure and \fref{fig01c06} shows decrease of BLE errors for this scanning round. Comparison between WiFi and BLE suggests that WiFi should have smaller error and better accuracy but there might be issues for both technologies in the middle of the building.

\section{Evaluation}\label{sec:Evaluation}
Evaluating of all the data has multiple approaches and combinations since in needs to consider information such as algorithm implemented or multiple device types and radio signal technologies. Since evaluation uses previously described WKNN algorithm, fist thing that needs to be decided is which $k$ variable will be used in this evaluation.

\subsection{Decide K values for WKNN}\label{sec:TestingKValuesForWKNN}

\subsection{Compare device technologies}\label{sec:CompareDeviceTechnologies}
This evaluation filters all the fingerprints based on device technology and tests them against each other. It is meant to show which device type is more effective and has more precise measurements.

\begin{table}[h]
	\begin{center}
		\begin{tabular}{ l C{2cm} C{2cm} C{2cm} | C{2cm} C{2cm} C{2cm} }
			\cline{2-7}
			& \multicolumn{3}{c|}{Mean error (m)} & \multicolumn{3}{c}{Max error (m)} \\
			\hline
			& BLE & WiFi & Combined & BLE & WiFi & Combined \\ 
			\hline	
			Wear & 2.27 & 2.67 & 2.21 & 12.90 & 11.11 & 10 \\		
			Mobile & 2.05 & 0.88 & 0.85 & 11.50 & 8 & 6.06 \\
			All devices & 2.25 & 1.77 & 1.52 & 12.90 & 11.11 & 10 \\
			\hline
		\end{tabular}
		\caption{Device comparison: mean and max errors (in meters)}
		\label{tab04c06}
	\end{center}
\end{table}

\tref{tab04c06} shows the comparison of device fingerprints and their calculated Mean and Max error values. BLE technology does not show much of a difference in devices, just in about 0.2 meters which makes them comparable. The WiFi, on the other hand, shows a big difference between these two types making mobile a better solution with over two times lower mean error. Finding the exact reason goes back to checking the data and figuring out that wear device does not contain 5 GHz WiFi receiver. Every manufacturer is in control of hardware for its devices and Huawei decided not to implement it due to high power usage. Records from 5 GHz are more precise than those of 2.4 GHz making this a reason why wear precision is not as good when comparing to mobile device.

%% Combined data wear drags down phone