\chapter{Conclusion}\label{sec:Conclusion}
This thesis has introduced a new way to collect radio fingerprints on mobile in combination with wear device to improve indoor stationary localization. This solution also included changes to data distribution between devices and the server. Created system consists of a server, mobile and wear devices with the Android operating system and WearOS for wear, both of them with Bluetooth Low Energy support. This system is designed to enable creation of radio fingerprint maps and update them anytime. Evaluation of this system was based on the Weighted K-Nearest Neighbors algorithm. This evaluation works only with specifically defined beacons and WiFi access-points to maintain equal environment for all fingerprints. Based on the acquired data in a real world scenario, the results of the localization were evaluated using WiFi, BLE and their combination with addition to using data from mobile, wear and their combination.

Evaluation was composed of three main algorithm implementations to figure out how to use data from multiple device types to improve overall localization accuracy. First test, using data from all device types separately showed lower accuracy of WiFi localization on wear, this was traced back to wear device not possessing the ability to scan for 5 GHz WiFi signals. This actually makes overall localization less accurate when combining the data together, where mean error increased from 0.85 to 1.52 meters, making it almost two times worse. Second test used one fingerprint per device type and averaged their location resulting in improvement of overall accuracy compared to previous evaluation but it is still not as precise and using single mobile device. Third and final evaluation combined data from multiple fingerprints at the same position and created using different device types, data from wear are cloned into data from mobile fingerprints. Best results were reached when combining fingerprint data from both devices into a single one with up to 6\% improvement of overall accuracy which should be even higher in locations not using 5 GHz WiFi networks or using multiple devices supporting 5 GHz WiFi signals. When this signal type is used and other devices do not support it, better accuracy can be reach by not including WiFi from these device in the evaluation.

It was proved that wear device can improve localization when used with the mobile but it is important to keep in mind that at this time smartwatches do not possess high battery life and cannot be used to scan for a long periods of time. Higher battery life could mean the possibility for manufacturers to implement 5 GHz WiFi receiver and effectively improve localization, bringing accuracy to the level of mobile device. Important thing to mention is that smartwatch localization can sometimes be preferred since it is tightly connected to a person.

\subsection{Future improvements}\label{sec:FutureImprovements}
This application is meant to be used in different environments and the first improvement should be to enable uploading of new maps and differentiating between locations, make it possible to change between them and load different fingerprints based on a specific location. Another likely improvement is making scanning more viable by supporting changes in its settings like length of the scan or lengths of all sub-scans. Final and very important improvement could be implementation of location calculation and display, where scanned data would be send to the server which would return calculated location to the application. This part would be implemented at the end of development process or it could be implemented in a separate solution to prevent over-compilation of existing application.