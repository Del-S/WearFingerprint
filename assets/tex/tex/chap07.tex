\chapter{Conclusion}\label{sec:Conclusion}
This thesis has introduced a new way to collect radio fingerprints on mobile and wear device, smartphone and smartwatch, to improve indoor stationary localization. This solution also included changes to data distribution between devices. The system consists of server, mobile and wear devices with the Android operating system which supports Bluetooth Low Energy. This system is designed to enable creation of radio-maps and update them anytime. Evaluation of this system was based on the Weighted K-Nearest Neighbors algorithm. This evaluation work only with specifically defined beacons and WiFi access-points to maintain equal environment for all fingerprints. Based on the data acquired in a real world scenario, the results of the localization were evaluated using WiFi, BLE and their combination with addition of data from mobile, wear and their combination.

Evaluation was composed of three main algorithm implementations to figure out how to combine data from multiple device types and improve overall localization accuracy. First, testing data from each device type separate showed lower accuracy of WiFi localization on wear, this was traced back to wear device not possessing the ability to scan for 5 GHz WiFi networks. This actually makes overall localization less accurate when combining the data together, where mean error increased from 0.85 to 1.52 meters. Second, testing one fingerprint per device type and averaging them improved overall accuracy compared to previous evaluation but it is still not as precise and using single mobile device. Third and final evaluation combined data from multiple fingerprints based on device type together.

Best results were reached when combining fingerprint data from both devices into single one with up to 6\% improvement of overall accuracy which should be even higher in locations not using 5 GHz WiFi networks. When this network is used and wear device does not support them, it is better to not include WiFi data from wear in the evaluation.

Wear device improves localization when used with the phone but it is important to keep in mind that at this time wear device type does not possess high battery life and cannot be used to scan for a long period of time. Higher battery life could also mean the possibility for manufacturers to implement 5 GHz WiFi receiver and effectively improve localization bringing accuracy to the level of mobile device.

\subsection{Future improvements}\label{sec:FutureImprovements}
This application is meant to be used in different environment and the first improvement should be to enable uploading of new maps and differentiating between locations, make it possible to change between them and load different fingerprints based on specific location. Another likely improvements is to make scanning more viable by changing its settings like length. Final and very important improvement could be implementation of location calculation and display where scanned data would be send to the server which would return calculated location to the application. This part would be implemented at the end of development process or it could be implemented in a separate solution to prevent over compilation of existing application.